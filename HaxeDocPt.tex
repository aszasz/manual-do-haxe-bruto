\documentclass{haxe}

% Translation specifics (avoiding changes to haxe.cls for now)
% language
\usepackage[T1]{fontenc}
\usepackage[main=brazil,english]{babel}
% custom commands
\newcommand\translationextra[1]{#1}
\newcommand\translatornote[1]{\footnote{N.~do T.: #1.}}
% adapted commands from the 'haxe' class
\renewcommand{\define}[4][Definição]
	{\begin{myshaded}\noindent\textbf{#1: #2}\par\nobreak\noindent\ignorespaces#4\label{#3}\end{myshaded}}

% todo-related
\usepackage[left=4.7cm, right=2cm, top=2cm, bottom=4.2cm]{geometry}
\usepackage[draft]{todonotes}
\reversemarginpar

% title (TODO: move this to class file once it looks good)

\renewcommand{\maketitle}{
   \begin{titlepage}
     \setcounter{page}{-1}
			\begin{center}
				~\\[3cm]
				\includegraphics[scale=1.25]{assets/logo.pdf}~\\[1cm]
				{\huge \bfseries Haxe 3 Manual}\\[7cm]
				Haxe Foundation\\
				\today
			\end{center}
   \end{titlepage}
}


\input{tikz}

% Conventions:

% run-time, compile-time
% Haxe, Haxelib (unless we are talking about the command itself)
% Haxe Standard Library, Haxe Compiler
% object-oriented

% code example width for ebooks: 47

\begin{document}
\title{Haxe 3 Manual}
\author{Haxe Foundation}
\date{\today}
\maketitle


\clearpage
\todototoc
\listoftodos
\clearpage

\clearpage
\tableofcontents
\clearpage

%commit do Manual original:  b6ae800cfd028b850bbe3a04889b1a7f04372082
\chapter{Introdução}
\label{introduction}
\state{NoContent}

\section{O que é o Haxe}
\label{introduction-what-is-haxe}

O Haxe é uma linguagem de programação de alto nível e um compilador de código aberto. Ele permite a compilação de programas, escritos em sintaxe similar ECMAScript, em programas de outras linguagens alvo\translatornote{nos referimos a essas linguagens doravante por targets}. Empregando o nível apropriado de abstração é possível manter uma única base de código que compila para múltiplos targets.

O Haxe é fortemente tipado, mas o sistema de tipagem pode ser subvertido quando preciso. Utilizando informações de tipos, o sistema de tipagem do Haxe pode detectar erros em tempo de compilação que só seriam percebidos em tempo de compilação na linguagem target. Além do mais, informações de tipos podem ser usadas pelos geradores dos targets para a geração de código otimizado e robusto.

Atualmente, existem nove targets suportados, que possibilitam diferentes casos de uso com diferentes sistemas 

\begin{center}
\begin{tabular}{| l | l | l |}
	\hline
	Nome & Formato de saída & Usos principais \\ \hline
	Javascript & Sourcecode & Browser, Desktop, Mobile, Server \\
	Neko & Bytecode & Desktop, Server \\
	PHP & Sourcecode & Server \\
	Python & Sourcecode & Desktop, Server \\
	C++ & Sourcecode & Desktop, Mobile, Server \\
	Actionscript 3 & Sourcecode & Browser, Desktop, Mobile \\
	Flash & Bytecode & Browser, Desktop, Mobile \\ 
	Java & Sourcecode & Desktop, Server \\
	C\# & Sourcecode & Desktop, Mobile, Server \\ \hline
\end{tabular}
\end{center}

O restante da \ref{introduction} dá uma breve visão geral de com o que um programa em Haxe se parece e como o Haxe evoluiu desde de sua criação em 2005.

\Fullref{types} introduz as sete espécies de tipos em Haxe e discute como eles interagem uns com os outros. A discussão continua em \Fullref{type-system}, onde funcionalidades como \emph{unificação}, \emph{parâmetros de tipo} e \emph{inferência de tipo} são explicadas.

\Fullref{class-field} é totalmente sobre as estruturas de classes do Haxe e, entre outros tópicos, lida com \emph{propriedades}, \emph{campos alinhados} e \emph{funções genéricas}.

Em \Fullref{expression} vemos como fazer que os programas realmente façam algo através do uso de \emph{expressões}.

\Fullref{expression} descreve algumas das funcionalidades do Haxe em detalhe, tais como \emph{localização de padrões}, \emph{interpolação de strings} e \emph{eliminação de código morto}. Isso conclui a referência para linguagem Haxe.

Continuamos com a referência para o compilador de Haxe, que primeiro trata do básico em \Fullref{compiler-usage} antes de prosseguir para a funcionalidades avançadas em \Fullref{cr-features}. Finalmente nos aventuramos no mundo estimulante das \emph{macros de Haxe} em \Fullref{macro} para ver como algumas tarefas comuns podem ser grandiosamente simplificadas.

No capítulo seguinte, \Fullref{std}, exploramos tipos importantes e conceitos da Biblioteca Padrão do Haxe. Nós então aprendemos sobre o gerenciador de pacotes Haxelib em \Fullref{haxelib}.

Os abstratos de Haxe afastam muitas diferenças entre os targets, mas algumas vezes é importante interagir com um target diretamente, o que é o assunto de\Fullref{target-details}.

\section{Sobre esse documento}
\label{introduction-about-this-document}

Esse documento é o manual oficial para o Haxe 3. Como tal, não é um tutorial para iniciantes e não ensina programação. Entretanto, os tópicos são grosseiramente concebidos para ser lidos em ordem e há referências para tópicos ``vistos anteriormente'' e a tópicos ``ainda por vir''. Em alguns casos, uma seção anterior faz uso de informação de uma seção adiante se isso simplifica a explicação. Essas referências são referenciadas adequadamente e não devem, normalmente, ser um problema a leitura adiantada sobre outros tópicos.

Utilizamos muito código fonte para manter uma conexão prática dos materiais teóricos. Esses exemplos de código são quase sempre programas completos que vem com uma função main que possa ser compilada ``como está''. Entretanto, algumas vezes unicamente as partes importantes são exibidas. O código fonte aparece como este:

\begin{lstlisting}
Código haxe aqui
\end{lstlisting}

Ocasionalmente, nós demonstramos como o código de Haxe é gerado, para o que usualmente exibimos o target \target{Javascript}.

Além do mais, definimos um conjunto de termos nesse documento. Predominantemente, isso é feito quando se introduz um novo tipo ou quando um termo é específico ao Haxe. Nós não definimos todo novo aspecto que introduzimos, e.g., o que é uma classe, para evitar o inchamento do texto.

\define{Nome definido}{define-definition}{Descrição da definição}

Em alguns lugares, esse documento tem caixas de \emph{trívia}. Essas incluem informações laterais, tais como: porque certas decisões foram tomadas durante o desenvolvimento do Haxe, ou: porque uma funcionalidade em particular foi mudada nas versões anteriores do Haxe. Essas informações são geralmente sem importantância e podem ser puladas uma vez que pretenderm trazer apenas trivialidades:

\trivia{Assunto da Trivia}{Informações históricas sobre o desenvolvimento da linguagem}

\section{Autores e contribuições}
\label{introduction-authors-and-contributions}

A maior parte do conteúdo desse documento foi escrita por Simon Krajewski enquanto trabalhando para a Haxe Foundation. Gostaríamos de agradecer essas pessoas por suas conntribuições:

\begin{itemize}
	\item Dan Korostelev: Conteúdo adicional e edição
	\item Caleb Harpre: Conteúdo adicional e edição
	\item Josefiene Pertosa: Edição
	\item Miha Lunar: Edição
	\item Nicolas Cannasse: Criador do Haxe
	\translationextra{\item Arthur Szász: Primeiro esforço de tradução para o Português}
\end{itemize}

\section{Hello World}
\label{introduction-hello-world}

O programa seguinte imprime "Hello World" depois de ser compilado e executado:

\haxe{assets/HelloWorld.hx}

Isso pode ser testado salvando o código acima em um arquivo chamado \ic{HelloWorld.hx} e chamando o compilador do Haxe assim: \ic{haxe -main HelloWorld --interp}. Ele então gera a seguinte saída: \ic{HelloWorld.hx:3 Hello world}. Há diverssas coisas para aprender disso:

\begin{itemize}
	\item Programas de Haxe são salvos em arquivos com uma extensão \ic{.hx}
	\item O compilador de haxe é uma ferramenta de linha de comando que pode ser chamada com parâmetros como \ic{-main} e \ic{--interp}
	\item Programas em haxe tem classes (\type{HelloWorld}, com maíuscula), que tem funções (\expr{main}, com minúscula).
	\item O nome do arquivo contendo a classe de Haxe main é o mesmo nome da próproa classe (nesse caso \type{HelloWorld.hx}).
\end{itemize}

\section{Histórico}
\label{introduction-haxe-history}
\state{Reviewed}

O projeto foi iniciado em 22 de outubro de 2005 pelo desenvolvedor francês \emph{Nicolas Canasse}, como um sucessor ao popular compilador de ActionScript2, de código aberto, \emph{MTASC} (Motion-Twin Action Script Compiler) e a sua própria linguagem \emph{MTypes}, que era uma experiência com a aplicação de inferência de tipos a uma linguagem orientada a objetos. A paixão de longa data de Nicolas pela concepção de linguagens de programação e o surgimento de novas oportunidades para juntar diferentes tecnologias como parte de seu trabalho desenvolvendo jogos na \emph{Motion-Twin} o levaram a criação de uma linguagem totalmente nova.

Escrita com X maiúsculo naquele tempo, a versão beta de haXe foi lançada em fevereiro de 2006, com os primeiros targets suportados sendo bytecode de AVM\footnote{Adobe Virtual Machine} e bytecode para \emph{Neko}\footnote{http://nekovm.org}, a maquina virtual do próprio Nicolas.

Nicolas Canasse, quem permanece como líder do projeto do Haxe até esta data, continuou melhorando o Haxe com uma visão clara, levando à subsequente divulgação de Haxe 1.0 em maio de 2006. Essa primeira versão maior veio com suporte a geração de código para \target{Javascript} e já possuia algumas das funcionalidades que definem o Haxe hoje, como a inferência de tipos e a subtipagem estrutural.

O Haxe 1 viu diversas modificações menores ao longo de dois anos, ganhando \target{Flash AVM2} como target junto com a ferramenta {haxelib} em agosto de 2006 e o target\target{Actionscript 3} em março de 2007. Durante esses meses, houve forte focalização na melhoria da estabilidade, do que resultaram diversas versões resolvendo pequenos bugs.

Haxe 2.0 foi divulgado em julho de 2008, incluindo o target \target{PHP}, cortesia de \emph{Franco Ponticelli}. Um esforço similar de \emph{Hugh Sanderson} levou a adição do target \target{C++} em julho de 2009 com a versão 2.04

Assim como o Haxe 1, o que seguiu foram diversas meses de versões para estabilidade. Em Janeiro de 2011 saiu a versão 2.07 com suporte a \emph{macros}. Por volta dessa época, \emph{Bruno Garcia} se juntou a equipe como mantenedor do target \target{Javascript}, que viu vastas melhorias nos lançamentos seguintes: 2.08 e 2.09.

Depois da versão 2.09, \emph{Simon Krajewski} se juntou ao time e o trabalho em direção ao Haxe 3 começou. Além disso, os targets \target{C\#} e \target{Java} de \emph{Cauê Waneck} acharam seus caminhos para dentro dos builds do Haxe. Decidiu-se, então, fazer uma versão final do Haxe 2, que aconteceu em julho de 2012, com a divulgação do Haxe 2.10.

No final de 2012, a chave do Haxe 3 foi virada e a equipe do Compilador Haxe, agora amparada pela recém-fundada \emph{Haxe Foundation}\footnote{http://haxe-foundation.org}, se focou nesta próxima grande versão. Haxe 3 foi subsequentemente lançado em maio de 2013.



\part{Language Reference}
% \input{02-types.tex}
% \input{03-type-system.tex}
% \input{04-class-field.tex}
% \input{05-expressions.tex}
% % commit original em 18/mar/2015 b7604e06222c7846f3fa4331d18741545db624dc:
\chapter{Funcionalidades da Linguagem}
\label{lf}
\state{NoContent}

\section{Compilação Condicional}
\label{lf-condition-compilation}

O Haxe permite compilação condicional com o uso de \expr{#if}, \expr{#elseif} e \expr{#else} e com a verificação de sinalizadores de compilação (compiler flags).

\definition{Sinalizador de compilação}{define-compiler-flag}{Um sinalizador (flag) de compilação é um valor configurável que pode influênciar o processo de compilação. Tais sinalizadores podem ser definidos pela chamada através da linha de comando com \expr{-D chave=valor} ou apenas \expr{-D chave}, caso em que o valor assume valor-padrão \expr{"1"}. O compilador também define diversos sinalizadores internamente para passar informação entre diferentes passos de compilação.}

O exemplo seguinte mostra um exemplo de uso de compilação condicional

\extratranslation{(N do T: Exemplo de chamada do compilador haxe ... -D debug debug_level=2)}

\haxe{assets/ConditionalCompilation.hx}

Compilando isso sem sinalizadores de compilação, deixaria apenas a linha \expr{trace("ok");} no corpo do método main. Os outros ramos são descartados durante o processamento do arquivo. Ainda assim, esses ramos devem conter sintaxe válida, mas o código não é verificado quanto a tipos.

As condições depois de \expr{#if} e \expr{#elseif} permitem as seguintes expressões:

\begin{itemize}
    \item Qualquer identificador é substituído pelo valor do sinalizador de compilação com o mesmo nome. Observe que \expr{-D flag-qualquer} a partir da linha de comando leva os sinalizadores \expr{flag-qualquer} e \expr{flag\_qualquer} a serem definidas.
    \item Os valores constantes de tipos \type{String}, \type{Int} e \type{Float} são usados diretamente.
    \item Os operadores booleanos \expr{\&\&} (e), \expr{||} (ou) e \expr{!} (não) funcionam como esperado.
    \item Os operadores \expr{==}, \expr{!=}, \expr{>},\expr{>=}, \expr{<}, \expr{<=} podem ser usados para comparar valores.
    \item Parenteses \expr{()} podem ser usados para agrupar expressões como de costume.
\end{itemize}

O analisador de sintaxe (parser) do Haxe não separa \expr{flag-qualquer} como um único símbolo (token) e ao invés disso o lê como um operador binário de subtração \expr{flag - qualquer} Em casos como esse a versão com sublinhado \expŕ{flag_qualquer} tem que ser usada.

\paragraph{Sinalizadores internos do Compilador}

Uma lista exaustiva de todas as definições internas pode ser obtida com a chamada do Compilador do Haxe com o argumento \expr{--help-defines}. O compilador permite múltiplos sinalizadores \expr{-D} por compilação.

Veja também a \tref{Lista de Sinalizadores de Compilação list}{lf-condition-compilation-flags}.
\subsection{Sinalizadores Globais do Compilador}
\label{lf-condition-compilation-flags}

A partir do Haxe 3.0, você pode obter a lista dos \tref{sinalizadores de compilação}{lf-condition-compilation} suportados ao rodar \expr{haxe --help-defines}

\begin{center}
\begin{tabular}{| l | l |}
	\hline
	\multicolumn{2}{|c|}{Global Compiler Flags} \\ \hline
	Flag &  Description \\ \hline
	\expr{absolute-path} &  Escreve o caminho absoluto para o arquivo no local de saída \\
	\expr{advanced-telemetry}  &  Permite que o SWF seja medido com a ferramenta Monocle  \\
	\expr{as3} &  Definido quando exportando código fonte flash9 as3 \\
	\expr{check-xml-proxy}  &  Verifica os campos usados do proxy xml  \\
    \expr{core-api}  &  Definido no contexto da api do cerne (api do core) \\
	\expr{cppia}  &  Gera o cpp instruction assembly experimental \\
	\expr{dce}  &  O atual modo de  \tref{Eliminação de Código Morto}{cr-dce} \\
	\expr{dce-debug}  &  Mostra o log da \tref{Eliminação de Código Morto}{cr-dce} \\
	\expr{debug}  &  Ativado quando compilando com \expr{-debug} \\
    \expr{display}  &  Ativado durante a \tref{completagem de texto}{cr-completion} \\
	\expr{dll-export}  &  Linkagem experimental de GenCPP \\
	\expr{dll-import}  &  Linkagem experimental de GenCPP \\
	\expr{doc-gen}  & Não aplica remoções/mudanças de forma a gerar a documentação corretamente \\
	\expr{dump}  &  Descarrega a AST tipada completa para debugagem interna \\
	\expr{dump-dependencies}  &  Descarrega as dependências das classes \\
	\expr{fdb}  &  Habilita informações completas do debug de flash ao FDB para debugagem interativa \\
	\expr{flash-strict}  &  Tipagem mais estrita para debugar o target flash \\
	\expr{flash-use-stage}  &  Mantém o estágio inicial da biblioteca SWF \\
	\expr{format-warning}  &  Escreve um warning para cada string formatado. para compatibildade com 2.x \\
	\expr{gencommon-debug}  &  Interno ao GenCommon \\
	\expr{haxe-boot}  &  Da o nome 'haxe' para a classe de boot do flash boot ao invés do nome gerado \\
	\expr{haxe-ver}  &  O valor atual da versão do Haxe \\
	\expr{hxcpp-api-level}  &  Fornecido para compatibilidade entre as versões hxcpp \\
	\expr{include-prefix}  &  Prefixa o caminho para os arquivos include gerados \\
	\expr{interp}  &  O código é compilado para ser executado com \expr{--interp} \\
	\expr{java-ver=[version:5-7]}  & Determina a versão de Java para ser usada de target \\
	\expr{js-classic}  &  Não usa uma função wrapper nem o modo strict na exportação de JS \\
	\expr{js-es5}  &  Gera JS para corridas de acordo com ES5 \\
	\expr{js-flatten}  &  Gera classes to use fewer object property lookups \\
	\expr{macro} & Definido quando compilamos código no \tref{contexto de macro}{macro} \\
	\expr{macro-times} & Exibe o tempo por macro quando usado com \expr{--times} \\
	\expr{neko-source} & Exporta fonte neko source ao invés de bytecode \\
	\expr{neko-v1} &  Mantém compatibilidade com  Neko 1.x \\
	\expr{net-target=<nome>}  &  Define o target .NET target. Assume o padrão para net. xbox, micro \_(Micro Framework\_, compact \_(Compact Framework)\_ são alguns valores válidos \\
	\expr{net-ver=<version:20-45>}  &  Define a versão de .NET a servir de target \\
	\expr{network-sandbox}  &  Usa a sandbox da rede local ao invés do acesso local de arquivo \\
	\expr{no-compilation}  &  Desabilita a compilação final de CPP \\
	\expr{no-copt}  &  Desabilita a otimização de completagem de texto \_(para debug)\_ \\
	\expr{no-debug}  &  Remove todas as macros de debug macros da saída de cpp \\
	\expr{no-deprecation-warnings} & Não vaisa se campos anotados com \expr{@:deprecated} são usados \\
	\expr{no-flash-override}  &  Muda a sobrescrição sobre algumas classes básicas para métodos com sufixo HX apenas no flash \\
	\expr{no-inline}  &  Desabilita \tref{inlining}{class-field-inline} \\
    \expr{no-macro-cache}  &  Desabilita o armazenamento de contexto de macros \\
	\expr{no-opt}  &  Desabilita otimizações \\
	\expr{no-pattern-matching}  &  Desabilita \tref{pattern matching}{lf-pattern-matching} \\
	\expr{no-root}  &  Interno ao GenCS  \\
	\expr{no-swf-compress}  &  Desabilita a compressão de saídas SWF \\
	\expr{no-traces}  &  Desabilita todas as chamadas a \expr{trace} \\
	\expr{php-prefix}  &  Compilado com \expr{--php-prefix} \\
	\expr{real-position}  &  Desabilidta o mapeamento de origem do haxe para o target C\# \\
	\expr{replace-files}  &  Interno ao GenCommon \\
	\expr{scriptable}  &  Interno ao GenCPP \\
	\expr{shallow-expose}  &  Expõe tipos ao escopo envolvente do limite do Haxe sem escrever no objeto window \\
	\expr{source-map-content}  &  Inclui as origens hx como parte do mapa fonte de JS \\
	\expr{swc}  &  Exporta um SWC ao invés de um SWF \\
	\expr{swf-compress-level=<level:1-9>}  &  Define o tanto de compressão para a saída SWF \\
    \expr{swf-debug-password=<yourPassword>}  &  Define uma senha para debugar. O campo senha é encriptado usando o algorimto MD5 e previne debugagem não autorizada do se swf. Sem esse sinalizador \expr{-D fdb} não usará senha \\
	\expr{swf-direct-blit}  &  Usa aceleração de hardware para processar gráficos com blit \\
	\expr{swf-gpu}  &  Usa funcionalidades de composição de  GPU ao desenhar gráficos \\
	\expr{swf-mark}  &  Interno ao GenSWF8 \\
	\expr{swf-metadata=<file.xml>}  &  Inclui o conteúdo de \expr{<file.xml>} como metadata no  swf. \\
	\expr{swf-preloader-frame}  &  Insere o primeiro frame vazio no swf. Para ser usado junto com \expr{-D flash-use-stage} e \expr{-swf-lib} \\
	\expr{swf-protected}  &  Compila o private do Haxe como protected no SWF ao invés de public \\
    \expr{swf-script-timeout}  &  Tempo máximo de processamento (em segundos) do ActionScript antes que a caixa de diálogo ``stuck'' seja exibida \\
    \expr{swf-use-doabc}  &  Usa o tag de swf \expr{DoAbc} ao invés de \expr{DoAbcDefine} \\
	\expr{sys}  &  Definido para todas as plataformas de sistema \\
	\expr{unsafe}  &  Permite código inseguro quando o target é  C\# \\
	\expr{use-nekoc}  &  Usa o compliador nekoc ao invés do interno \\
	\expr{use-rtti-doc}  &  Permite acesso a documentação durante a compilação \\
	\expr{vcproj}  &  Interno ao GenCPP \\
\end{tabular}
\end{center}

\section{Externs}
\label{lf-externs}

Externs podem ser usados para descrever interações específicas com um target com tipagem segura. Eles são definidos como classes normais, exceto que:

\begin{itemize}
    \item a palavra-chave \expr{class} é precedida pela palavra-chave \expr{extern}.
    \item \tref{métodos}{class-field-method} não tem expressões e
    \item todos os tipos dos argumentos e do retorno são explícitos.
\end{itemize}

Um exemplo normal da \tref{Biblioteca Padrão do Haxe}{std} é a classe Math, como o seguinte extrato mostra:

\begin{lstlisting}
extern class Math
{
	static var PI(default,null) : Float;
	static function floor(v:Float):Int;
}
\end{lstlisting}

Vemos que a externs podem definir tanto métodos quanto variáveis (na verdade, \expr{PI} é declarado como uma \tref{propriedade}{class-field-property} somente-leitura). Uma vez que essa informação está disponível para o compilador, ela permite acessos a campo de acordo com isso e também conhece os tipos:

\haxe{assets/Extern.hx}

Isso funciona porque o tipo do retorno do método \expr{floor} é declarado como \type{Int}. 

A Biblioteca Padrão do Haxe vem com muitas externalizações para os targets \target{Javascript} e \target{Flash}. Elas permitem o acesso às APIs nativas de uma maneira segura em relação aos tipos e são ferramentas úteis para a concepção de APIs de alto-nível. Também há externalizações para muitas bibliotecas populares na \tref{haxelib}{haxelib}.

Os \target{Flash}, \target{Java} e \target{C\#} permitem a inclusão direta de bibliotecas nativas a partir da \tref{linha de comando}{compiler-usage}. Detalhes específicos são explicados nas respectivas seções \Fullref{target-details}.

Alguns targets como \target{Python} ou \target{JavaScript} podem exigir o ``import'' de código adicional que carregue uma classe \expr{extern} de um módulo nativo. O Haxe oferce maneiras de declarar essas dependências também descritas nas seções respectivas de \Fullref{target-details}.

\paragraph{Argumentos Rest e escolha de tipos}
\since{3.2.0}

O pacote haxe.extern provê dois tipos que ajudam o mapeamento da semântica nativa ao Haxe:

\begin{description}
	\item[\type{Rest<T>}:] Esse tipo pode ser usado como um argumento final de função para permitir a passagem de um número arbitrário de argumentos de chamada adicionais. O parâmetro de tipo que pode ser usado para restringir esses argumentos a um tipo específico.
	\item[\type{EitherType<T1,T2>}:] Esse tipo permite o uso de qualquer dos seus parâmetros de tipo, representando assim uma escolha de tipos. Ele pode ser aninhado para permitir mais do que dois tipos diferentes de tipos.
\end{description}

Demonstramos o uso neste exemplo de código:

\haxe{assets/RestAndEitherType.hx}

\section{Estensão Estática}
\label{lf-static-extension}

\define{Extensão estática}{define-static-extension}{Uma extensão estática permite uma pseudo-extensão dos tipos existentes sem modificar sua fonte. No Haxe isso é conseguido através da declaração de um método estático com um primeiro argumento do tipo que se pretende estender e então trazendo para a definição da classe (a ser estendida) para o contexto através de \expr{using}.

Extensões estáticas podem ser uma ferramenta poderosa que permite aumentar os tipos sem verdadeiramente mudá-los. O exemplo seguinte demonstra o uso:

\haxe{ptassets/StaticExtension.hx}

Claramente, \type{Int} não oferece um método nativo \expr{triplo}, ainda assim esse programa compila e dá o resultado \expr{36}, como esperado. Isso porque a chamada para \expr{12.triplo()} é transformada em \expr{IntExtender.triplo(12)}. Existem três requerimentos para isso:

\begin{enumerate}
    \item Tanto o literal \expr{12} e o primeiro argumento de \expr{triplo} são do tipo \type{Int}.
    \item A classe \type{IntExtender} é trazida para o contexto através de \expr{using Main.IntExtender} 
    \item \type{Int} não tem um campo \expr{triplo} por si só (se tivesse, esse campo tomaria a prioridade sobre a extensão estática).
\end{enumerate}

Extensões estáticas são geralmente considerados adoçantes sintáticos (syntactic sugar) e, de fato, são; vale apena observar que elas podem ter um efeito dramático na legibilidade do código: Ao invés de chamadas aninhadas no formato \expr{f1(f2(f3(f4(x))))}, o encadeamento de chamada na forma \expr{x.f4().f3().f2().f1()} pode ser usado.

Seguindo as regras previamente descritas em \Fullref{type-system-resolution-order}, múltiplas expressões \expr{using} são verificadas de baixo para cima, com os tipos de cada módulo, bem como com os campos de cada tipo sendo verificados de cima para baixo. A utilização de um módulo (em oposição a um tipo específico de um módulo, ver\Fullref{type-system-modules-and-paths}) como uma extensão estática traz todos os seus tipos ao contexto.

\subsection{Na Biblioteca Padrão do Haxe}
\label{lf-static-extension-in-std}

Diversas classes na Biblioteca Padrão do Haxe são adequadas para o uso de extensões estáticas. O próximo exemplo mostra o uso de \type{StringTools}:

\haxe{assets/StaticExtension2.hx}

Ainda que \type{String} não tenha uma funcionalidade \expr{replace} por si próprio, a extensão estática \expr{using StringTools} fornece um. Como normalmente, o resultado em \target{Javascript} esclarece a transformação:

\begin{lstlisting}
Main.main = function() {
	StringTools.replace("adc","d","b");
}
\end{lstlisting}

As seguintes classes da Biblioteca padrão são concebidas para ser usadas como extensões estáticas:

\begin{description}
    \item[\type{StringTools}:] Fornece funcionalidades estendidas sobre os strings, como substituição ou eliminação de espaços.
    \item[\type{Lambda}:] Fornece métodos funcionais sobre iteráveis.
    \item[\type{haxe.EnumTools}:] Fornece funcionalidades sobre a informação de tipos em enums e em suas instâncias.
    \item[\type{haxe.macro.Tools}:] Fornece diferentes extensões para trabalhar com macros (ver \Fullref{macro-tools}).
\end{description}
        
\trivia{using ``using''}{Como a palavra \expr{using} foi adicionada a linguagem, se tornou comum em inglês aparecerem problemas na utilização da expresão ``usando using'' (``using using''), razão pela qual o autor do manual optou por chamar a funcionalidade por sua definição formal: Extensão Estática}

\section{Correspondência de padrões (Pattern Matching)}
\label{lf-pattern-matching}
\state{NoContent}

\subsection{Introdução}
\label{lf-pattern-matching-introduction}

Pattern matching é o processo de ramificação dependendo da correspodência de um dado valor a padrões, possivelmente complicados. No Haxe, toda correspondência de padrões é feita dentro de uma expressão \tref{\expr{switch} expression}{expression-switch} onde as expressões \expr{case} individuais representam os padrões. Aqui exploraremos a sintaxe de diferentes padrões usando essa estrutura de dados como o exemplo corrente:

\haxe[firstline=1,lastline=4]{ptassets/PatternMatching.hx}

\extratranslation{Tree = árvore, Leaf = folha, Node = Nó; l por left:esquerda, r por right:direita}

O básico da correspondência de padrões inclui:

\begin{itemize}
    \item Padrões sempre serão correspondidos de cima para baixo.
    \item O mais alto dos padrões que casar com o valor de entrada terá sua expressão executada.
    \item Um padrão \expr{_} corresponde com qualquer coisa, então \expr{case _}: é igual a \expr{default:}
\end{itemize}

\subsection{Correspondência de Enum}
\label{lf-pattern-matching-enums}

Enums podem ser correspondidos por seus constructors de uma maneira natural:

\haxe[firstline=8,lastline=26]{ptassets/PatternMatching.hx}


A correspondência de nós será feita de cima para baixo e escolherá a primeira alternativa em que houver o casamento. A interpretação passo a passo ajuda a entender o processo:

case Leaf(_): correspondência não bate, porque myTree é um nó

case Node(_,Leaf(_)): falha porque a sub-árvore a direita não é uma folha

case Node(_, Node(Leaf("bar"),_)); correspondência corresponde.

case _: não chega a ser testada, porque a anterior já resolveu.


\subsection{Captura de  variável} 

É possível achar qualquer vaolor de um sub-padrão achando sua correpondência com um identificador:

var myTree = Node(Leaf("foo"),Node(Leaf("bar"), Leaf("foobar")));
var name = switch(myTree) {
    case Leaf(s): s;
    case Node(Leaf(s), _): s;
    case _: "nenhum";
}
trace(name); // foo

Isso retornaria algum dentre os seguintes:

  - Se myTree é uma única folha, o nome da folha é retornado

  - Se myTree é um nó cuja subárvore esquerda é uma folha, o nome da folha da subárvore da esquerda é retornado (o caso do exemplo retornando "foo").

  - Em outros casos "nenhum" é retornado.

Também é possível usar = para capturar valores que são correspondentes a um padrão:

var node = switch(myTree) {
    case Node(leafNode = Leaf("foo"), _):leafNode;
    case x: x;
}
trace(node); // Leaf(foo)

Aqui, leafNode é associado a Leaf("foo") se o dado de entrada corresponder a isso. Em todos os outros casos, a própria myTree é retornada: case x funciona de forma similar a case _ achando correspondência com qualquer coisa, mas colocando um identificador com um nome como x, o valor de x (qualquer que seja) vai ser associado a (var) node.

\section{Correspondência de Estrutura}

Também é possível achar a correspondência contra padrões de campos de estruturas anônimas e instâncias:

var myStructure = {
    name: "haxe",
    rating: "awesome"
};
var value = switch(myStructure) {
    case { name: "haxe", rating: "poor" }: throw false;
    case { rating: "awesome", name: n }:n;
    case _: "no awesome language found";
}
trace(value); // haxe

No segundo caso nós associamos o campo name ao identificador n, caso rating corresponda a "awesome". É claro que essa estrutura poderia ser colocada no formato Tree do exemplo anterior para combinar a correspondência entre estrutura e enum.

Uma limitação em relação a instâncias de classe é que você não pode achar a correspondêncoa com campos ou sua classe pai.

\subsection{Correspondência de array}

Arrays com tamanho fixo  podem ser correspondidos:

var myArray = [1, 6];
var match = switch(myArray) {
    case [2, _]: "0";
    case [_, 6]: "1";
    case []: "2";
    case [_, _, _]: "3";
    case _: "4";
}
trace(match); // 1

Isso imprimirá 1 porque array[1] corresponde a 6 e array[0] pode ser qualquer coisa.

\subsection{Padrões ou}

O operador | pode ser usado em qualquer lugar dentro dos padrões para descrever múltiplos termos aceitáveis:

var match = switch(7) {
   case 4|1: "0";
   case 6|7: "1";
   case 2: "2";
}
trace match; // 1

Se há uma variável capturada em um padrão ou, ela deve aparecer nos dois subpadrões.

\subsection{Guardas}

Também é possível restringir ainda mais os padrões com a sintaxe case;;; if(condition):

var myArray = [7, 6];
var s = switch(myArray) {
    case [a, b] if (b > a): b + ">" +a;
    case [a, b]: b + "<=" +a;
    case _: "found something else";
}
trace(s); // 6<=7

O primeiro case tem uma condição adicional de guarda if(b>a). Esse caso só será selecionado se a condição for verdadeira, senão a busca por correspondência continua com o próximo case.

\subsection{Correspondência sobre múltiplos valores}

A sintaxe de array pode ser usada para fazer a correspondência sobre múltiplos valores:

var s = switch [1, false, "foo"] {
    case [1, false, "bar"]: "0";
    case [_, true, _]: "1";
    case [_, false, _]: "2";
}
trace(s); // 2

Isso é bastante similar à correspondência normal de arrays, mas há algumas diferenças:

  - O número de elementos é fixo, então padrões de diferentes tamanhos de array não serão aceitos.

  - Não é possível capturar o valor do switch em uma variável, i.e. case x não é permitido aqui (case _ ainda é).

\subsection{Extratores}

Desde Haxe 3.1.0

Extratores permitem a aplicação de transformações aos valores a serem correspondidos. Isso é frequentemente útil quando uma pequena operação é exigida em um valor correspondido antes que o processo de correspondência prossiga:

enum Test {
   TString(s:String);
   TInt(i:Int);
}

class Main {
     static public function main() {
          var e = TString("fOo");
          switch(e) {
              case TString(temp):
              switch(temp.toLowerCase()) {
                   case "foo": true;
                   case _: false;
              }    
              case _: false;
          }
     }
}

Aqui nós temos que capturar o valor do constructor do enum TString em uma variável tem e usar um switch aninhado sobre temp.toLowerCase() (toLowerCase=paraMinúscula). Obviamente, queremos que a correspondência tenha sucesso se TString tiver o valor "foo" sem consideração ao uso de maiúsculas ou minúsculas. Isso pode ser simplificado com extratores:
(N. do T.: temp é o caso de x, sempre será verdadeiro, verificar...)

enum Test {
    TString(s:String);
    TInt(i:Int);
}
class Main {
    static public function main() {
    var e = TString("fOo");
    var success = switch(e) {
        case TString(_.toLowerCase() => "foo"): true;
        case _: false;
        }  
    }
}

Extratores são identidficados pela expresãoExtratora => expressão a corresponder. O compilador gerá código que é similar ao exemplo anterior, mas a sintaxe original foi bastante simplificada. Extratores consistem de duas partes, que são separadas pelo operador =>:

1. O lado esquerdo pode ser qualquer expressão, onde todas as ocorrências de _ são substituídas pelo valor que atualmente detém a correspondência.

2. O lado direito é um padrão que é comparado contra o resultado da valoração do lado esquerdo.

Uma vez que o lado direito é um padrão, ele pode conter um outro extrator. O exemplo seguinte encadeia dois extratores:

class Main {
    static public function main() {
        switch(3) {
            case add(_, 1) => mul(_, 3) => a: trace(a);
        }
    }
    static function add(i1:Int, i2:Int) {
        return i1 + i2;
    }

    static function mul(i1:Int, i2:Int) {
        return i1 * i2;
    }
}

Isso imprime 12 como o resultado da chamada a add(3, 1) onde 3 é o valor que correspondeu, e mul (4, 3) onde 4 é o resultado da chammada a add. Vale a pena observar que a do lado direito do segundo operador => é uma variável capturada (ver seção)

Atualmente não é possível usar extratores dentro de padrões or:

class Main {
    static public function main() {
        switch("foo") {
            // Extractors in or patterns are not allowed
            case (_.toLowerCase() => "foo") | "bar":
        }
    }
}

Entretanto, é possível ter padrões ou do lado direito de um extrator. então o exemplo anterior compilaria sem os parenteses.

\subsection{Verificações de exaustividade}

O compilador assegura que nenhum caso possível seja esquecido:

switch(true) {
    case false:
} // Verdadeiro não verificado

O tipo correspondente admite dois valores true e false, mas apenas falso é verificado.

\subsection{Verificações de padrão inúteis}

De forma similar, o compilador detecta padrões que nunca corresponderão ao valor de entrada:

switch(Leaf("foo")) {
    case Leaf(_) 
    | Leaf("foo"): // Esse padrão não é usado
    case Node(l,r):
    case _: // Esse padrão não é usado
}

\section {Interpolação de strings}

Com o Haxe 3 não é mai necessário concatenar partes de um string devido a introdução da interpolação de strings. Identificadores especiais, denotados pelo cifrão $ dentro de um string que utilize aspas simples ', são processados como se fossem identificadores concatenados

var x = 12;
trace(’O valor de x é $x’);n // O valor de x é 12

Além disso é possível incluir expressões inteiras em um string utilizando ${expr}, onde expr é qualquer expressão válida do Haxe.

var x = 12;
trace(’A soma de $x e 3 é ${x + 3}’);
// A soma de 12 e 3 é 15

A interpolação de strings é uma funcionalidade de compilação e não tem qualquer impacto em tempo de execução. O exemplo acima é equivalente a concatenação manual, que é exatamente o que o compilador gera:


Claro que o uso de strings com aspas simples sem qualquer interpolação permanece válido, mas cudiado deve ser tomado em relação ao caractere $, uma vez que ele dispara a interpolação. Se um cifrão de fato desejar ser utilizado no string, $$ pode ser usado.

trace("A soma de " + x + " e 3 é " + (x + 3));

\trivia{Interpolação de strings antes do Haxe 3}{A interpolação de strings é uma funcionalidade desde a versão 2.09. Naquele momento, a macro Std.format tinha de ser usada, sendo tanto mais lenta quanto menos confortável que a nova sintaxe de interpolação de strings}

\subsection{Preenchimento de arrays}

O preenchimento de arrays em Haxe usa a sintaxe existente para permitir a inicialização concisa de arrays. Éla é identificada pelos constructs for ou while

class Main {
    static public function main() {
        var a = [for (i in 0...10) i];
        trace(a); // [0,1,2,3,4,5,6,7,8,9]
        var i = 0;
        var b = [while(i < 10) i++];
        trace(b); // [0,1,2,3,4,5,6,7,8,9]
    }
}

A variável a é inicializada para um array contendo os números de 0 a 9. O compilador gera código que adiciona o valor de cada iteração do loop no array, o que é equivalente ao seguinte código:

var a = [];
for (i in 0...10) a.push(i);

A variável b é inicializada para um array com os mesmo valores, mas atráves de um estilo diferente de compressão usando while ao invés de for. Novamente, o código seguinte seria equivalente

var i = 0;
var a = [];
while (i<10) a.push(i++);

A expressão do laço pode ser qualquer uma, incluindo condições e laços aninhados, então o seguinte funciona como esperado:

class AdvArrayComprehension {
    static public function main() {
        var a = [
            for (a in 1...11)
                for(b in 2...4)
                    if (a % b == 0) 
                         a+ "/" +b
        ];
        trace(a); // [2/2,3/3,4/2,6/2,6/3,8/2,9/3,10/2]
    }    
}  

\section{Iteradores}

Com o Haxe, é muito fácil definir iteradores personalizados e tipos de datas iteráveis. Esses conceitos são representados pelos tipos Iterator<T> e Iterable<T> respectivamente:

typedef Iterator<T> = {
    function hasNext() : Bool;
    function next() : T;
}
typedef Iterable<T> = {
    function iterator() : Iterator<T>;
}

Qualquer classe (ver seção) que unifique (ver seção) estruturalmente com umdesses tipos pode ser iterada usando um laço for (ver seção). Isso é, se a classe define métodos hasNext e next com tipos de retorno coincidentes, ela é considerada um iterador; se ela define um método iterador que retorna um Iterator<T> ela é considerada um tipo iterável.

class MyStringIterator {
    var s:String;
    var i:Int;
    public function new(s:String) {
        this.s = s;
        i = 0;
    }
    public function hasNext() {
        return i < s.length;
    }

    public function next() {
        return s.charAt(i++);
    }
}

class Main {
    static public function main() {
        var myIt = new MyStringIterator("string");
        for (chr in myIt) {
            trace(chr);
        }
    }
}

O tipo MyStringIterator nesse exemplo qualifica um iterador: Ele define um método hasNext que retorna Bool e um método next que retorna String, tornando o compátivel com Iterator<String>. O método main o instancia e então itera sobre ele.

class MyArrayWrap<T> {
    var a:Array<T>;
    public function new(a:Array<T>) {
        this.a = a;
    }

    public function iterator() {
        return a.iterator();
    }
} 

class Main {
    static public function main() {
        var myWrap = new MyArrayWrap([1, 2, 3]);
        for (elt in myWrap) {
            trace(elt);
        }
    }  
}

Aqui nós não definimos um iterador completo como no exemplo anterior, ao invés disso definimos que MyArrayWrap<T> tem um método iterator, encaminhando efetivamente o método iterador do tipo envelopado Array<T>.

\section{Amarração de funções}

O Haxe 3 permite 3 formas de amarração de funções com argumentos parcialmente aplicados. Cada tipo de função pode ser considerar a existência de uma campo de amarração bind, que pode ser chamado com o número desejado de argumentos de forma a criar uma nova função. Isso é demontrado aqui:

class Bind {
    static public function main() {
        var map = new Map<Int,String>();
        var f = map.set.bind(_, "12");
        $type(map.set); // Int -> String -> Void
        $type(f); // Int -> Void
        f(1);
        f(2);
        f(3);
        trace(map); // {1 => 12, 2 => 12, 3 => 12}
   }
}

A linha 4 amarra a função map.set a uma variável chamada f e aplica 12 como um segundo argumento. O caractere sublinhado _ é usado para denotar que esse argumento não é amarrado, o que é mostrado comparando os tipos de map.set e f: O argumento amarrado String é efetivamente cortado do tipo, tornandoo tipo Int->String->Void em Int->Void.

Uma chamada de f(1) chama realmente map.set(1,"12"), as chamadas para f(2) e f(3) são análogas. A última linha prova que os três índices são de faot mapeados para o valor "12".

O sublinhado _ pode ser omitido para os argumentos que vem depois dele, se o primeiro argumento fosse amarrado através de map.set.bind(1), disponibilizaríamos uma função String->Void que definiria o  valor para o índice 1 na chamada de f.

\Trivia{Retrochamada}{Antes do Haxe 3, o Haxe costumava conhecer uma palavra-chave callback que poderia ser chamada com um argumento de função seguido por um número de argumentos de amarração. O nome se originou de um uso comum onde uma função de retrochamada (callback) é creada com o objeto this sendo amarrado. Callback  permitiria a amarração de argumentos apenas da esquerda para a direita pois não havia suporte para o caractere _. A escolha pelo uso de _ foi controversa e muitas outras sugestões foram feitas, nenhuma delas considerada superior. Afinal, o sublinhado _ ao menos parece que está dizendo "preencha o valor aqui", o que descreve a semântica de forma agradável.}

\section{Metadados}

Diversos constructs podem ser atribuídos com metadados personalizados:

 - declarações de classe e enums

 - campos de classes

 - Constructors de enums

 - Expressões

Essas informações de metadados podem ser obtidas em tempo de execução através da API haxe.rtti.Meta:

import haxe.rtti.Meta;

@author("Nicolas")
@debug

class MyClass {
    @range(1, 8)
    var value:Int;

    @broken
    @:noCompletion
    static function method() { }
}
class Main {
    static public function main() {
        trace(Meta.getType(MyClass)); 
                      // { author : ["Nicolas"], debug : null }
    trace(Meta.getFields(MyClass).value.range); // [1,8]
    trace(Meta.getStatics(MyClass).method); // { broken: null }
    }
}

Nós podemos facilmente identificar metadados pelo caractere inicial @, seguido pelo nome do metadado e, opcionalmente por um número de argumentos constantes separados por vírgulas fechados por parentes.

  - A classe MyClass tem um metadado author com um único argumento "Nicolas", bem como um metadado debug sem argumentos.

  - A variável membro value tem um metadado range com dois argumentos do tipo Int; os argumentos são 1 e 8.

  - O método estático method tem um metadado broken ser argumentos, bem como um metadado :noCompletion também sem argumentos.

  O método main acessa os valores desses metadados usando a API. O output revela a estrutura dos dados obtidos:

  - Há um campo para cada metadado, com o nome do campo sendo o nome do metadado.

  - Os valores de campo são os argumentos do metadado. Se não há argumentos, o valor do campo é null. De outra forma o valor é um array, com um elemento por argumento.

  - Metadados que começam por dois pontos : são omitidos. Esse tipo de dado é conhecido por metadado de compilação.

Os valores permitidos para argumentos de metadados são:

- Constantes (ver seção)

- Declaração de arrays (ver seção) (se todos os seus elementos forem qualificados corretamente)

- Declaração de objetos (ver seção) (se todos os seus elementos forem qualificados)

\section{Controle de Acessos}

O controle de acessos pode se4r usado se as opções de visibilidade básica não são suficientes. Ele é aplicável em nível de classe e em nível de campos e entende duas diretivas:

Permissão de acesso: Uma dada classe ou campo da classe tem o seu acesso garantido ao alvo usando o metadado :allow(target) (ver seção).

Acesso forçado: Um alvo é forçado a permitir o acesso para a classe ou campo usando o metadado :acess(target) (ver seção).

Nesse contexto, um alvo pode ser um caminho separado por pontos (dot-path) para:

  - um campo de classe,

  - uma classe ou tipo abstrato, ou

  - um pacote.

Se for uma classe ou um tipo abstrato, a modificação de acesso se estende a todos os campos daquele tipo. Da mesma forma, se for um pacote, a modificação de acesoo se estende a todos os tipos daquele pacote e recursivamente a todos os campos desses tipos

@:allow(Main)
class MyClass {
    static private var foo: Int;
}

class Main {
    static public function main() {
        MyClass.foo;
    }  
}

Aqui, MyClass.foo pode ser acessado do método main porque MyClass recebe a anotação @:allow(Main). Isso também funcionaria com @:allow(Main.main) e ambas as versões poderiam alternativamente ser anotadas ao campo foo ao invés da classe MyClass

class MyClass {
    @:allow(Main.main)
    static private var foo: Int;
}
class Main {
    static public function main() {
        MyClass.foo;
    } 
}

Se um tipo não pode ser modificado para permitir essa espécie de acesso, o método de acesso dever forçar o acesso:

class MyClass {
    static private var foo: Int;
}
class Main {
    @:acess(MyClass.foo)
    static public function main() {
        MyClass.foo;
    }
}

A anotação @:acess(MyClass.foo) efetivamente subverte a visibilidade do campo foo dentro do método main.

\trivia{Sobre a escolhad de metadados}{A funcionalidade da linguagem de controle de acessos usa a sintaxe de metadados ao invés de sintaxe específica da linguagem. Há diversas razões para isso:

  - Sintaxe adicional frequentemente adiciona complexidade ao processamento de separação de palavras da linguagem (parsing) e também adiciona (muitas) novas palavras-chave.

  - Sintaxe adicional exige aprendizado adicional pelo usuário da linguagem, enquanto a sintaxe de metadados é algo que já é conhecido.

  - A sintaxe de metadados é flexível o bastante para permitir a extensão dessa funcionalidade.

  - Os metadados podem ser acessados/gerados/modificados pelas macros de Haxe.

É claro, o principal retrocesso do uso da sintaxe de metadados é que você não recebe um relatório de erros no caso de digitar errado tanto a chave do metadado (@:acesss por exemplo) ou o nome da classe/pacote. Entretanto, com essa funcionalidade você poderá obter um erro quando você tentar acessar um campo privado que você não tem acesso, portanto não há a possibilidade de erros silenciosos.

Desde Haxe 3.1.0

Se o acesso é permitido a uma interface (ver seção), ele se extende a todas as classes implementando aquela interface:


class MyClass {
    @:allow(I)
    static private var foo: Int;
}
interface I { }
class Main implements I {
    static public function main() {
    MyClass.foo;
    }
}

Isso também funciona para o acesso garantido a classes pai, caso em que se estende à todas classes descendentes.

\trivia{Funcionalidade quebrada}{A extensão de acesso à classes descendentes e classes implementadoreas deveria funcionar no Haxe 3.0 e foi, inclusive documentada dessa forma. Enquanto se escrevia esse manual se descobriu que essa parte da implementação de controle de acesso simplesmente estava faltando.}

\section{Constructors alinhados}

Desde Haxe 3.1.0

Se um constructor é declarado para ser alinhado (ver subseção), o compilador poderá tentar otimizá-lo por fora em certas situações. Há diversos requerimentos para isso funcionar:

  - O resultado da chamada do constructor deve ser diretamente atribuído a uma variável local.

  - A expressão do campo constructor deve conter apenas atribuições a seus campos.

O exemplo seguinte demonstra o alinhamento de constructor:

class Point {
    public var x:Float;
    public var y:Float;
    
    public inline function
    new(x:Float, y:Float) {
        this.x = x;
        this.y = y;
    }
}

class Main {
    static public function main() {
        var pt = new Point(1.2, 9.3);
    }
}

A observação do exposto em Javascript revela o efeito:

Main.main = function() {
    var pt_x = 1.2;
    var pt_y = 9.3;
};

\section{Remoting}

Parte II - Compiler Referenca



\part{Compiler Reference}
% \input{07-compiler-usage.tex}
% % commit do original em 17/mar 8b2b2f8b6494ddb0d74a8f137088d7d1a505cdb6
\chapter{Funcionalidades do compilador}
\label{cr-features}
\state{NoContent}

\section{Metadados de compilação}
\label{cr-metadata}

Partindo do Haxe 3.0, você pode conseguir a lista dos metadados definidos pelo compilador executando \expr{haxe --help-metas}

\begin{center}
\begin{tabular}{| l | l | l |}
	\hline
	\multicolumn{3}{|c|}{Global metatags} \\ \hline
	Metatag &  Descrição  &  Plataforma \\ \hline
	@:abstract &  Determina a classe subjacente de implementação  como \tref{tipo abstrato}{types-abstract}  &  cs  java \\
	@:access \_(Target path)\_  &   Força o acesso privado ao tipo ou campo do pacote, ver \tref{controle de acesso}{lf-access-control}  &  todas \\
	@:allow \_(Target path)\_  &  Permite acesso privado do tipo ou campo do pacote, ver \tref{controle de acesso}{lf-access-control}  &  todas \\
	@:annotation  &  Definições de anotações (\expr{@interface}) no imports de \expr{-java-lib} serão anotadas com esse metadado. Não tem efeito sobre tipos compilados pelo Haxe &  java \\
    @:arrayAccess  &  Permite \tref{accesso como array}{types-abstract-array-access} em um (tipo) abstrato  &  todas \\
	@:autoBuild \_(Build macro call)\_  &   Estende o metadado \expr{@:build} para todas as classes implementadoras ou exetensoras, ver  \tref{autobuild macro}{macro-auto-build}  &  todas \\
	@:bind  &  Sobrescreve a declaração de classe Swf  &  flash \\
	@:bitmap \_(Bitmap file path)\_  &  \_Embarca os bitmaps dados na classe (deve estender \expr{flash.display.BitmapData})   &  flash \\
	@:build \_(Build macro call)\_  &   Monta uma classe ou enum a partir de uma macro, ver \tref{montagem de tipos}{macro-type-building}  &  todas \\
	@:buildXml  &    &  cpp \\
	@:classCode  &  Usada para injetar código nativo a plataforma em uma classe &  cs  java \\
	@:commutative  &  Declara um operador de abstrato como comutativo  &  todas \\
	@:compilerGenerated  &  Marca um campo como gerado pelo compilador. Não deveria ser usado pelo usuário final &  cs  java \\
	@:coreApi &  Identifica essa classe como uma classe do cerne da API (força a verificação da API)  &  todas \\
    @:coreType  &  Identica um tipo abstrato como {pertencente ao cerne (\tref{core type}{types-abstract-core-type}) de forma que não exija implementação  &  todas \\
	@:cppFileCode  &     &  cpp \\
	@:cppNamespaceCode  &    &  cpp \\
    @:dce  &  Força a \tref{eliminção de código morto (DCE)}{cr-dce} mesmo quando \expr{-dce full} não é especificado  &  todas \\
	@:debug  &  Força a geração de informação de debug a ser gerada no Swf mesmo sem \expr{-debug}   &  flash \\
	@:decl   &     &  cpp \\
	@:defParam  &    &  all \\
	@:delegate  &  Automaticamente inserida por \expr{-net-lib} nos delegados   &  cs \\
	@:depend  &     &  cpp \\
	@:deprecated   &  Automaticamente inserida por  \expr{-java-lib} nos campos de classe anotados com a anotação \expr{@Deprecated}. Não tem efeito em timpos compilados pelo Haxe  &  java \\
	@:event  &  Automaticamente inseridos por \expr{-net-lib} sobre eventos. Não tem efeitos sobre tipos compilados pelo Haxe   &  cs \\
	@:enum  &  Define conjuntos finitos de valores nas definições de abstratos, ver \tref{abstratos de enums}{types-abstract-enum}  &  todas \\
	@:expose \_(?Name=Class path)\_  &  Torna a classe disponível no objeto \expr{window}  ou em \expr{exports} para node.js, ver \tref{exposing Haxe classes for Javascript}{target-javascript-expose} &  js \\
	@:extern  &  Marca o campo como um extern para que não seja gerado  &  all \\
	@:fakeEnum \_(Type name)\_  &  Trata o enum como uma  coleção de valores do tipo especificado  &  all \\
	@:file(File path)  &  Inclui um dado arquivo binário no target Swf e o associa com a classe  (deve estender \expr{flash.utils.ByteArray})  &  flash \\
	@:final  &  Previne uma class classe de ser estendida  &  todas \\
    @:font \_(TTF path Range String)\_  &  Embarca uma fonte TrueType dada na classe (deve estender \expr{flash.text.Font})  &  flash \\
	@:forward \_(lista of nome de campo)\_  &  \tref{Encaminha o acesso de campo}{types-abstract-forward} ao typo subjacente  &  todas \\
    @:from   &  Specifica que o campo do abstrato é uma operação cast (moldagem) a partir do tipo identificado na função, ver \tref{casts implícitos}{types-abstract-implicit-casts}  &  todas \\
	@:functionCode  &     &  cpp \\
	@:functionTailCode  &    &  cpp \\
	@:generic &  Marca uma classe ou campo de classe como \tref{generic}{type-system-generic} para que cada combinação de parâmetro de tipo gere seu próprio tipo/campo  &  todas \\
	@:genericBuild  & Monta instâncias de um tipo usando a macro especificada   &  todas \\
	@:getter \_(nome do campo de classe)\_  &  Gera uma função 'getter' nativa para o campo dado &  flash \\
	@:hack   &  Permite estender classes marcadas como \expr{@:final}  &  todas \\
	@:headerClassCode  &    &  cpp \\
	@:headerCode   &     &  cpp \\
	@:headerNamespaceCode  &    &  cpp \\
	@:hxGen  &  Anota que uma classe extern foi gerada pelo Haxe  &  cs  java \\
    @:ifFeature \_(funcionalidade (feature))\_  &  Faz com que um campo seja mantido pelo \tref{DCE}{cr-dce} se a funcionalidade é dada como parte da compilação  &  todas \\
	@:include &     &  cpp \\
	@:initPackage  &    &  all \\
	@:internal  &  Gera o campo/classe anotada com acesso \expr{internal}  &  cs  java \\
	@:isVar  &  Força um campo físico a ser gerado para propriedades que de outra forma não o exigiriam  &  todas \\
	@:keep   &  Faz com que um campo ou tipo seja mantido \tref{DCE}{cr-dce}  &  all \\
	@:keepInit  &  Faz com que uma classe seja mantida pela \tref{DCE}{cr-dce} mesmo que todos os seus campos sejam removidos  &  todas \\
	@:keepSub &  Estende o metadado \expr{@:keep} para todas as classes implementadoras ou estensoras  &  todas \\
	@:macro  &  \_(deprecated)\_  &  todas \\
	@:meta   &  Internamente usada para marcar um campo de classe como sendo o campo metadado &  todas \\
	@:multiType \_(parâmetros de tipo relevantes)\_  &  Especifica que um abstrato escolhe seu this-type a partir de suas funções \expr{@:to}  &  todas \\
	@:native \_(caminho do tipo de saída)\_  &  Reescreve o caminho de uma classe ou enum durante a geração  &  todas \\
	@:nativeGen  &  Annotates that a type should be treated as if it were an extern definition - platform native  &  cs  java \\
	@:noCompletion  &  Previne o compilador de sugerir o \tref{completamento}{cr-completion} sobre esse campo  &  todas \\
	@:noDebug &  Não gera informação de debug para o Swf mesmo se \expr{-debug} está habilitado   &  flash \\
	@:noDoc  &  Previne um tipo de ser incluído na geração automática de documentação &  todas \\
	@:noImportGlobal  &  Previne um campo estático de ser importado como \expr{import Class.*}  &  todas \\
	@:noPackageRestrict  &  Permite a um módulo ser acessado de todos os targets se localizado em seu primeiro tipo  &  todas \\
	@:noStack &     &  cpp \\
	@:noUsing &  Previne um campo de ser usado com \expr{using}  &  todas \\
	@:notNull &  Declara que um tipo abstrato não aceita \tref{valores \expr{null}}{types-nullability}  &  all \\
	@:ns  &  Internamente usado pelo gerador de Swf para manusear namespaces   &  flash \\
    @:op \_(operação)\_  &   Declara um campo abstrato como um \tref{sobrescritor de operador}{types-abstract-operator-overloading}  &  todas \\
	@:optional  &  Marca o campo de uma estrutura como opcional, ver \tref{argumentos opcionais}{types-nullability-optional-arguments}  &  todas \\
	@:overload \_(especification de função)\_  &  Permite que o campo seja chamado com diferentes tipos de argumento. A especificação de função não pode ser uma expressão  &  todas \\
	@:privateAccess  &  Permite o acesso  privado a tudo para a expressãp anotada  &  todas \\
	@:property  &  Marca um campo de propriedade para ser compilado como uma propriedade nativa de C\# property   &  cs \\
	@:protected  &  Marca um campo de classe como sendo protegido  &  todas \\
	@:public  &  Marca um campo de classe como sendo público  &  todas \\
	@:publicFields  &  Força todos os campos de classe das classes herdeiras a serem públicos & todas \\
	@:readOnly  &  Gera um campo com a palavra-chave \expr{readonly} nativa &  cs \\
	@:remove  &  Faz com que uma interface seja removida de todas as classes implementadoras antes da geração  &  todas \\
    @:require \_(flag de compilação)\_  &  Permite acesso a um campo apenas se a {flag de compilação}{lf-condition-compilation} especificada está habilitada  &  all \\
	@:rtti   &  Insere informações de tipo de tempo de execução, ver \tref{RTTI}{cr-rtti}  &  all \\
	@:runtime  &    &  todas \\
	@:runtimeValue  &  Marks an abstract as being a runtime value  &  all \\
	@:setter \_(nome do campo de classe)\_  &  Gera uma função setter nativa para o dado campo &  flash \\
	@:sound \_(caminho do arquivo)\_  &  Inclui um arquivo \_.wav\_ or \_.mp3\_ dado ao target Swf e o associa com a classe (deve estender \expr{flash.media.Sound})  &  flash \\
	@:struct  &  Marca uma definição de classe como um struct  &  cs \\
	@:suppressWarnings  &  Insere uma anotação SuppressWarnings para a classe Java gerada  &  java \\
	@:throws \_(Type as String)\_  &  Insere uma declaração \expr{throws} à função gerada  &  java \\
    @:to  &  Especifica que o campo do tipo abstrato (abstract) é uma operação cast (fusão) para o tipo identificado na função, ver \tref{casts implícitos}{types-abstract-implicit-casts} & todas \\
    @:transient  &  Adiciona o sinalizador (flag) \expr{transient} ao campo de classe  &  java \\
	@:unbound  &  Interno ao compilador para denotar variável global unbounded &  all \\
	@:unifyMinDynamic  &  Permite a uma coleção de tipos que se unifique com Dynamic  &  todas \\
	@:unreflective  &    &  cpp \\
	@:unsafe  &  Declara uma classe pu um método com a flag \expr{unsafe} de C\#   &  cs \\
	@:usage  &    &  todas \\
	@:volatile  &    &  cs  java \\
\end{tabular}
\end{center}

\section{Eliminação de código morto}
\label{cr-dce}

A eliminação de código morto ou \emph{DCE} (abreviando Dead Code Elimination) é uma funcionalidade que remove código não utilizado da saída. Depois de tipar, o compilador avalia os pontos de entrada (geralmente o método main) e recursivamente determina que tipos e campos são usados. Os campos utilizados são marcados adequadamente e os campos não marcados são removidos de suas classes.

A DCE tem três modos que são definidos quando chamados pela linha de comando:

\begin{description}
    \item[dce std:] Apenas as classes na Biblioteca padrão são afetadas pelo DCE. Isso é o modo padrão para todos os targets.
    \item[-dce no:] Nenhuma eliminação de código morto é executada.
    \item[dce full:] Todas as classes são afetadas pelo DCE.
\end{description}
O algoritmo de DCE funciona bem com código tipado, mas pode falhar quando se usa  \tref{dynamic}{types-dynamic} e\tref{reflection}{std-reflection}. Isso pode exigir a marcação explícita de campos ou classes como sendo utilizadas, ao atribuir lhes os seguintes metadados:

\begin{description}
    \item[\expr{@:keep}:] Se for usado em uma classe, a classe e todos os seus campos não serão afetados pelo DCE.
    \item[\expr{@:keepSub}:] Se usado em uma classe, funciona com \exor{@:keep} na classe anotada e em todas as suas subclasses.
    \item[\expr{@:keepInit}:] Usualmente, uma classe que teve tosos os seus campos removidos pelo DCE (ou é vazia mesmo) é removida do resultado. Ao usar esse metadado, classes vazias são mantidas.
\end{description}

Se uma classe precisa ser marcada com \expr{@:keep} da linha de comando ao invés da edição de seu código fonte, há uma macro de compilação disponível para fazer isso: \expr{--macro keep('type dot path')}. Veja a href{http://api.haxe.org/haxe/macro/Compiler.html#keep}{APIhaxe.macro.Compiler.keep} para detalhes dessa macro. Ela irá marcar o pacote, módulo ou subtipo a ser mantido pelo DCE e incluí-los na compilação.

O compilador automaticamente define um sinalizador (flag) de compilação \expr{dce} com valor entre \expr{ "std"}, \expr{"no"} ou \expr{"full"} dependendo do modo ativado. Isso pode ser usado epara a \tref{compilação condicional}{lf-condition-compilation}.

\trivia{Revisão do DCE}{O DCE foi originalmente implementado no Haxe 2.07. Essa implementação foi considerada uma função a ser usada quando o código era explicitamente tipado. O problema com ela era que diversas funcionalidades, as mais importantes sendo as interfaces, faziam com que todos os campos de classe a serem tipados de forma a verificar a segurança-de-tipos. Isso subverteu o DCE efetivamente todo, demandando sua revisão para o Haxe 2.10}

\trivia{DCE e try.haxe.org}{O DCE para \type{Javascript} viu vastas melhorias quando o website \url{http://try.haxe.org} foi divulgado. A recepção inicial do código gerado em \target{Javascript} foi variada, levando a uma seleção mais delicada de qual código deveria ser eliminado}

\section{Completamento de texto}
\label{cr-completion}

Por conta do grande número de funcionalidades do Haxe e sua poderosa inferência de tipos, \href{http://haxe.org/documentation/introduction/editors-and-ides.html}{IDEs e editores de código} não podem lidar facilmente com o completamento de texto analisando sintaticamente os arquivos de Haxe. O Haxe oferece apoio ao completamento de expressões para IDEs construídos diretamente dentro do compilador graças ao parâmetro de linha de comando \ic{--display}

Vamos dar uma olhada no seguinte exemplo:
\begin{lstlisting}
class Test {
    public static function main() {
        trace("Hello".|
    }
}
\end{lstlisting}

A barra vertical indica a posição do cursor depois que o ponto foi premido. Neste momento, é o trabalho da IDE chamar o compilador do Haxe com os parâmetros usuais de compilação, além de um adicional \ic{--display Test.hx@73}.

Esse parâmetro determina o arquivo no qual nós queremos algum completamento de expressões e a posição em bytes (não em caractetes) do cursor no arquivo. No exemplo dado, você deve obter 73 se você contar caracteres com o CRLF de fins de linha do Windows 

O Haxe desempenhará toda a análise sintática e tipagem como faria para a compilação normal, exceto que não gerará qualquer código. Se ele chega no byte 73 no arquivo \ic{Test.hx} durante a compilação, ele expõe a informação sobre o tipo da expressão à esquerda do ponto.

No nosso caso, isso seria \type{String}. A exceção sendo que no caso de um objeto anônimo ou uma instância de classe, o compilador irá, ao invés, listar os diferentes campos em um XML e imprimí-los para uma possível saída. 


\begin{lstlisting}
<list>
<i n="length"><t>Int</t><d>...</d></i>
<i n="charAt"><t>index : Int -> String</t><d>...</d></i>
<i n="charCodeAt"><t>index : Int -> Int</t><d>...</d></i>
<i n="indexOf">
  <t>value : String -> ?startIndex : Int -> Int</t>
  <d>...</d>
</i>
...
</list>
\end{lstlisting}

Nesse caso, todos os métodos e campos públicos de \type{String} são listados. Essa informação pode ser usada diretamente pela IDE para fornecer dicas de código e completamento de texto.

\paragraph{Completamento de funções}
Esse mecanismo de completamento funciona tanto com ponto quanto com parenteses de abertura, para que você possa conseguir informações de tipos sobre listas de campos e tipos de argumento de chamada de funções:

\begin{lstlisting}
class Test {
    public static function main() {
        trace("Hello".split(|
    }
}
\end{lstlisting}

Isso lhe dara o seguinte resultado: \expr{<type>delimiter : String -> Array<String></type>} \_(por favor, observe que o conteúdo do tipo não contém caracteres de escape de html-escaped na saída verdadeira)\_

\paragraph{Completamento de pacotes}
Ele também funciona para pacotes, escaneando os arquivos .hx disponíveis no caminho-de-classe:

\begin{lstlisting}
import haxe.|
Retornará :

<list>
    <i n="BaseCode"><t></t><d></d></i>
    <i n="EnumFlags"><t></t><d></d></i>
    ...
    <i n="io"><t></t><d></d></i>
    <i n="macro"><t></t><d></d></i>
    <i n="remoting"><t></t><d></d></i>
    ...
</list>
\end{lstlisting}

\paragraph{Manuseio de erros}
Quando executado em modo de completamento de texto, o compilador não exibirá erros, mas ao invés disso, tentará ignorá-los ou se recuperar a partir deles.

Se um erro sem recuperação ocorre enquanto se obtém a completagem de texto, o compilador de Haxe imprimirá a mensagem de erro ao invés da saída de completagem. Você pode, então, tratar qualquer saída não XML como uma mensagem de erro que impede a completagem.

\paragraph{Servidor de cache de compilação}
\since{2.09}

Para conseguir a melhor velocidade tanto para compilação quanto para completagem, você pode usar o parâmetro de linha de comando \ic{--wait} para iniciar um servidor de compilação Haxe; Você também pode usar \ic{-v} para fazer o servidor imprimir um log. Aqui está um exemplo:

\begin{lstlisting}
haxe -v --wait 6000
\end{lstlisting}

Você pode, então, concectar ao servidor Haxe, enviar parâmetros de linha de comando seguidos por um byte 0 e, então, ler a resposta (sejam resultados de completagem de texto ou erros)

Use o parâmetro de linha de comando \ic{--connect} para fazer com que o Haxe envie seus comandos de compilação para o servidor ao invés de executá-los diretamente:

\begin{lstlisting}
haxe --connect 6000 myproject.hxml
\end{lstlisting}

Por favor, observe que você pode usar o parâmetro \ic{--cwd} na primeira linha de comando enviada para mudar o diretório atual de trabalho do servidor de Haxe. Usualmente caminhos de classe e outros arquivos são relativos ao seu projeto.

\paragraph{Como ele funciona}
O servidor de compilação fará cache das seguintes coisas:

\begin{description}
    \item[arquivos analisados sintaticamente (``parseados'')] os arquivos só serão ``parseados'' novamente se eles forem modificados ou se havia um erro de análise
    \item[chamadas a haxelib] os resultados anteriores das chamadas a haxelib serão reutilizados(apenas para completagem: eles serão ignorados quando se faz uma compilação
	\item[módulos tipados] módilos de compilação serão armazenados depois de uma compilação com sucesso e podem ser reutilizados depois em compilações/completagem se nenhuma de suas dependências tiverem sido modificadas
\end{description}

Voê pode conseguir leituras precisas dos tempos gastos pelo compilador e como o servidor de compilação os afetam inserindo \ic{--times} na linha de comando. 

\paragraph{Protocolo}
Como o próximo exemplo de Haxe/Neko mostra, você pode simplesmente conectar na porta do servidor e mandar todos os comandos (um por linha) terminando com um caractere binário 0. Você pode, então, ler os resultados

Macros de outros comandos podem fazer logs de eventos que não são erros. Da linha de comando, nós podemos ver a diferença entre o que é impresso em \ic{stdout} e o que é impresso em \ic{stderr}. Esse não é o caso no modo de socket. De forma a diferenciar entre os dois, mensagens de log (não erros) são prefixados com um caracter \ic{\\x01} e todos os caracteres de newline na mensagem são repostos pelo mesmo caractere \ic{\\x01}.

Warnings e outras mensagens também podem ser considerados erros, mas não erros fatais. Se um erro fatal ocorreu, uma única linha de mensagem \ic{\\x02} será enviada.

Aqui está algum código que se conectará ao servidor e lidará com detalhes do protocolo:

\begin{lstlisting}
class Test {
    static function main() {
		var newline = "\textbackslash\ n";
        var s = new neko.net.Socket();
        s.connect(new neko.net.Host("127.0.0.1"),6000);
        s.write("--cwd /my/project" + newline);
        s.write("myproject.hxml" + newline);
        s.write("\textbackslash\ 000");
		
        var hasError = false;
        for (line in s.read().split(newline))
		{
            switch (line.charCodeAt(0)) {
				case 0x01: 
					neko.Lib.print(line.substr(1).split("\textbackslash\ x01").join(newline));
				case 0x02: 
					hasError = true;
				default: 
					neko.io.File.stderr().writeString(line + newline);
            }
		}
        if (hasError) neko.Sys.exit(1);
    }
}
\end{lstlisting}

\paragraph{Efeito em macros}
O servidor de compilação pode ter alguns efeitos colaterais na \tref{execução de macros}{macro}.

\subsection{Acesso a campos}
\label{cr-completion-field-access}

\subsection{Chamada de argumentos}
\label{cr-completion-call-arguments}

\subsection{Uso}
\label{cr-completion-usage}

\subsection{Posição}
\label{cr-completion-position}

\subsection{Nível superior}
\label{cr-completion-toplevel}

\section{Recursos}
\label{cr-resources}
\flag{fold}{true}

O Haxe oferece um sistema de embarcação de recursos simples que pode ser usado par embarcar arquivos diretamente na aplicação compilada.

Enquanto pode não ser otimizado embarcar coisas grandes como imagens ou música no arquivo da aplicação, isso vem a ser conveniente para embarcar pequenos recursos como configurações ou dados XML.

\subsection{Embarcação(embedding) de recursos}
\label{cr-resources-embed}

Arquivos externos podem ser embarcados usando o argumento de compilação \emph{-resource}

\todo{what to use for listing of non-haxe code like hxml?}
\begin{lstlisting}
-resource hello_message.txt@welcome
\end{lstlisting}

O string depois do símbolo \emph{@} é o \emph{identificador do recurso} que é usado no código para acolher o recurso. Se for omitido (bem como o \emph{@}) o nome do arquivo se tornará o identificador do recurso.

\subsection{Acolhimento de recursos de texto}
\label{cr-resources-getString}

Para acolher o conteúdo de um recurso embarcado, utilizamos o método estático \emph{getString} de \type{haxe.Resource} e passando lhe um \emph{identificador de recursos}:

\haxe{assets/ResourceGetString.hx}

O código acima ira exibir o conteúdo do arquivo \emph{hello_message.txt} que incluímos anteriormente usando wellcome como identificador

\subsection{Acolhimento de recursos binários}
\label{cr-resources-getBytes}

Ainda que não seja recomendado embarcar grandes arquivos binários na aplicação, ainda pode ser útil embarcar dados binários. A representação binária de um recurso embarcado pode ser acessada com o método estático \emph{getBytes} de \type{Haxe.Resource}:

\haxe{assets/ResourceGetBytes.hx}

O tipo do retorno do método \emph{getBytes} é \type{haxe.io.Bytes}, que é um objeto que fornece acesso a bytes individuais dos dados.

\subsection{Detalhes de implementação}
\label{cr-resources-impl}

O Haxe usa a embarcação de recursos nativa da plataforma target se houver uma, se não fornece sua própria implementação.

\begin{itemize}
    \item Recursos em \emph{Flash} são embarcados como definições de ByteArray
    \item Recursos em \emph{C#} são incluídos no assembly compilado.
    \item Recursos em \emph{Java} são empacotados na arquivo JAR resultante
    \item Recursos em \emph{C++} são armazenados nas constantes de array de bites globais.
    \item Recursos em \emph{Javascript} são seriados no formato de serialização do Haxe e armazenados no campo estático da classe \type{haxe.Resource.
    \item Recursos em \emph{Neko} são armazenados como strings em um campo estático da classe \type{haxe.Resource}
\end{itemize}



\section{Informação de tipos em tempo de execução}
\label{cr-rtti}

O compilador do haxe gera informação de tipos em tempo de execução (RTTI - runtime type information) para classes que são marcadas com o metadado \expr{:rtti}  ou para classes que estendam classes com essa marcação. Essa informação é armazenada como um string XML em um campo estático \expr{__rtti} e pode ser processado através da \type{haxe.rtti.XmlParser}. A estrutura resultante é descrita em  \Fullref{cr-rtti-structure}.

\since{3.2.0}

O tipo \type{haxe.rtti.Rtti} foi introduzido de forma a simplificar o trabalho com RTTI. A extração dessa informação é agora muito fácil:

\haxe{assets/RttiUsage.hx}

\subsection{A estrutura da informação de tipos em tempo de execução}
\label{cr-rtti-structure}

\paragraph{Informação geral de tipos}

\begin{description}
    \item[path:] O \tref{caminho de tipo}{define-type-path} para o tipo
    \item[module:] O caminho de tipo do \tref{módulo}{define-module}que contém o tipo
    \item[file:] O caminho com barras (\slash) completo para o arquivo .hx que vonté o tipo. Esse pode ser \expr{null} no caso de não existir o arquivo, e.g., se o tipo é definindo através de uma \tref{macro}{macro}.
    \item[params:] Um array de strings representando os nomes dos \tref{parâmetros de tipo}{type-system-type-parameters} que o tipo tem. A partir do Haxe 3.2.0, isso não inclui as \tref{restrições}{type-system-type-parameter-constraints}.
    \item[ doc:] A documentação do tipo. Essa informação só é disponível se o \tref{sinalizador de compilação}{define-compiler-flag} \expr{-D use_rtti_doc} estiver em ação. De outra forma, ou se o tipo não tiver documentação, o valor é \expr{null}
    \item[isPrivate:] Se o tipo é ou não \tref{privado}{define-private-type}.
    \item[platforms:] Uma lista de strings representando os targets onde o tipo está disponível
     \item[meta:] Os metadados com que o tipo foi marcado.
\end{description}
  
\paragraph{Informaçao de tipo para classes}
\label{cr-rtti-class-type-information}

\begin{description}
    \item[isExtern:] Se a classe é ou não \tref{externalizada}{lf-externs}.
    \item[isInterface:] Se a classe é ou não é uma \tref{interface}{types-interfaces}.
    \item[superClass:] A classe pai da classe defininda por seu caminho de tipo e lista de parâmetros de tipo.
    \item[interfaces:] A lista de interfaces definidas por seu caminho de tipo e lista de parâmetros de tipo
    \item[fields:] A lista de seus \tref{campos de classe }{class-field} membros descritas em\Fullref{cr-rtti-class-field-information}.
    \item[statics:] A lista de campos de classe estáticos, descrita em \Fullref{cr-rtti-class-field-information}.
    \item[tdynamic:] O tipo que está \tref{implementado dinamicamente}{types-dynamic-implemented} pela classe, ou \expr{null} se o tipo não existe.
\end{description}

\paragraph{Informação de tipo enum}

begin{description}
    \item[isExtern:] Se o enum é ou não \tref{extern}{lf-externs}.
    \item[constructors:] A lista dos constructors do enum
\end{description}

\paragraph{Informações para tipo abstratos}

\begin{description}
    \item[to:] um array contendo \tref{casts implicitos}{types-abstract-implicit-casts} para qual o tipo se converte.
    \item[from:] um array contendo \tref{casts implícitos}{types-abstract-implicit-casts} a partir do qual o tipo se converte. 
    \item[impl:] A \tref{informação de tipo de classe}{cr-rtti-class-type-information} da classe de implementação.
    \item[athis:] O \tref{tipo subjacente}{define-underlying-type} do abstrato.
\end{description}

	
\paragraph{Informações para campos de clase}
\label{cr-rtti-class-field-information}

\begin{description}
    \item[name:] O nome do campo.
    \item[type:] o tipo do campo
    \item[isPublic:] Se o campo é ou não \tref{público}{class-field-visibility}.
    \item[isOverride:] Se o campo \tref{sobrescreve}{class-field-override} ou não outro campo.
    \item[doc:] A documentação do campo. Essa informação só é disponível se o  \tref{sinalizador de compilação}{define-compiler-flag} \expr{-D use_rtti_doc} estiver em ação. De outra forma, ou se o tipo não tiver documentação, o valor é \expr{null}.
    \item[get:] O \tref{comportamento de acesso de leitura}{define-read-access} do campo.
    \item[set:] O \tref{comportamento de acesso à escrita}{define-write-access} no campo.
    \item[params:] um array de strings representando os nomes dos \tref{parâmetros de tipo}{type-system-type-parameters} que o campo tem . A partir do Haxe 3.2.0, isso não inclui as \tref{restrições }{type-system-type-parameter-constraints}.
    \item[platforms:] Uma lista de strings representando os targets onde o campo está disponível.
    \item[meta:] Os metadados com os quais o campo foi marcado.
    \item[line:] O número da linha onde o campo é definido. Essa informação só está disponível se o campo tem uma expressão. De outra forma é \expr{null}.
    \item[overloads:] Uma lista dos overloads disponíveis para os campos, ou \expr{null} se não existe overload.
\end{description}

\paragraph{Informações para constructors de enum}
\label{cr-rtti-enum-constructor-information}

\begin{description}
    \item[name:] O nome do constructor.
    \item[args:] a lista dos argumentos que o constructor tem, ou \expr{null} se não há argumentos disponíveis.
    \item[doc:] A documentação do constructor. Essa informação só está disponível se o \tref{sinalizador de compilação}{define-compiler-flag} \expr{-D use_rtti_doc} estiver em ação. De outra forma, ou se o tipo não tiver documentação, o valor é \expr{null}
    \item[platforms:] Uma lista de strings representando os targets onde o campo está disponível.
    \item[meta:] Os metadados com os quais o constructor foi marcado.
\end{description}

% % commit original em 30/mar/2015: 69f35be313a4c3abf2a24b2c1df19846025baeb3
\chapter{Macros}
\label{macro}

Macros são, sem nenhuma dúvida, a funcionalidade mais avançada do Haxe. Elas frequentemente são percebidas como uma magia sombria que apenas alguns eleitos são capazes de dominar, ainda que não exista nada de mágico(e certamente nada de sombrio) sobre elas.

\define{Arvore Abstrata de Sintaxe (AST para  Abstract Sintax Tree)}{define-ast}{A AST resulta do processamento da separação das palavras (\emph{parsing}) do código fonte em uma estrutura tipada. Essa estrutura é exposta para as macros através dos tipos definidos no arquivo haxe/macro/Expr.hx da Biblioteca Padrão do Haxe}

\input{assets/tikz/macro-compilation-role.tex}

Um macro básica é uma \emph{transformação sintática}. Ela recebe zero ou mais \tref{expressões}{expression} e também retorna uma expressão. Se uma macro é chamada, ela efetivamente insere código no local de onde foi chamada. Com relação a isso, ela poderia ser comparada a um preprocessador como \expr{#define} in C++, mas uma macro de Haxe não é uma ferramenta de substituição de texto.

Nós podemos certamente identificar diferentes espécies de macros, que são executadas em diferentes estágios da compilação:

\begin{description}
    \item[Macros de Inicialização:] Essas são fornecidas pela linha de comando usando o parâmetro de compilação \ic{--macro}. Elas são executadas depois que os argumentos de compilação foram processados e o \emph{contexto para tipificação} já foi criado, mas antes que qualquer tipificação seja feita (\Fullref{macro-initialization}).
    \item[Macros de Build (Montagem)]: Essas são definidas para classes, enums e abstratos através do\tref{metadado}{lf-metadata}\expre{@:build} ou \expr{@:autobuild}. Elas são executadas por tipo, depois que o tipo foi definido (incluindo sua relação com outro tipos, como herança para classes) mas antes que os campos sejam tipados (Ver\Fullref{macro-type-building}).
    \item[Macros de expressão:] Essa são funções normais que são executadas tão logo sejam tipadas.
\end{description}


\section{Contexto de macros}
\label{macro-context}

\define{Contexto de macro}{define-macro-context}{O contexto de macor é o ambiente no qual a macro é executada. Dependendo do tipo de macro, ele pode ser considerado como uma classe em montagem ou como uma função sendo tipificada. Informações de contexto podem ser obtidas através da API \ic{haxe.macro.Context}.}

Macros de Haxe tem acesso a diferentes informações contextuais dependendo do tipo da macro. Além de acessar tais informações, o contexto também permite algumas modificaçãos tais como a definir um novo tipo ou registrar certas retrochamadas (callback). É importante entender que nem toda informação está disponível para todas as espécies de macro, como os exemplos seguintes demonstram:

\begin{itemize}
    \item Macros de inicialização vão achar que os métodos \expr{Context.getLocal*()} retornam \expr{null}. Não há tipo ou método local no contexto de uma macro de inicialização.
    \item Apenas macros de build recebem um valor adequado de \expr{Context.getBuildFields()}. Não há campos em montagem para as outras espécies de macro.
    \item Macros de build tem um tipo local (se incompletas), mas não tem métodos locais, então \expr{Context.getLocalMethod} retorna {null}.
\end{itemize}

A API de contexto é complementada pela API \expr{haxe.macro.Compiler} detalhada em \Fullref{macro-initialization}. Enquanto essa API está disponível para todas espécies de macros, cuidado deve ser tomado para qualquer modificação fora das macros de inicialização. Isso deriva da limitação natural da ordem de montagem não definida (ver seção), o que pode levar, por exemplo, a definição de um sinalizador (flag) através de Compiler.define() a ter efeito antes ou depois de uma verificação de \tref{compilação condicional}{lf-condition-compilation} aquele sinalizador.

\section{Argumentos}
\label{macro-arguments}

Na maior parte do tempo, argumentos para macros são expressões representadas como uma instância de uma expressão de enum (enum \type{Expr}). Como tal, elas são separadas sintaticamente, mas não são tipadas, implicando que elas podem ser qualquer coisa que se conforme com as regras de sintaxe do Haxe. A macor pode então inspecionar suas estruturas, ou (tentar) conseguir seu tipo usando \expr{haxe.macro.Context.typeof()}

É importante entender que argumentos para macros não tem garantia de serem valorados, então qualquer efeito colateral não tem garantia de ocorrer. Por outro lado, é importante entender que uma expressão como argumento pode ser duplicada por uma macro e usada multiplas vezes na expressão de retorno:

\haxe{assets/MacroArguments.hx}

A macro \expr{add} é chamada com argumento \expr{x++} e dessa forma retorna x++ + x++ usando a consolidação de expressão (ver seção), fazendo com que x seja incrementado duas vezes.

\subsection{ExprOf}
\label{macro-ExprOf}

Uma vez que \type{Expr} é compatível com qualquer entrada possível, o Haxe fornece o tipo \type{haxe.macro.ExprOf<T>}. Na maioria das vezes, esse tipo é idêntico a \type{Expr}, mas ele permite restringir o tipo das expressões aceitas. Isso é útilquando se combinam macros com \tref{extensões estáticas}{if-static-extension}:

\haxe{assets/ExprOf.hx}

As duas chamadas diretas para \expr{identity} são aceitas, mesmo que o argumento seja declarado como \expr{ExprOf<String>}. Pode ser surpreendente que \type{Int} \expr{1} seja aceito, mas é uma consequência lógica do que foi explicado sobre \tref{argumentos de macros}{macro-arguments}: As expressões de argumento nunca são tipadas, então não é possível para o compilador verificar sua compatibilidade através da \tref{unificação}{type-system-unification}. 

Isso é diferente para as duas próximas linhas que estão usando extensões estáticas (observe o \expr{using Main}): Para essas duas é mandatória a tipagem do lado esquerdo (\expr{"foo"} e \expr{1}) primeiro, de forma a dar sentido para o acesso ao campo \expr{identity}. Isso torna possível verificar tipos em relação aos tipos do argumento, o que faz com que \expr{1.identity()} não considerar \expr{Main.identity()} como um campo adequado.

\subsection{Constantes expressões}
\label{macro-constant-arguments}

Uma macro pode ser declarada para esperar \tref{constantes}{expression-constants} como argumentos:

\haxe{assets/MacroArgumentsConst.hx}

Com isso não é necessário dar a volta em expressões uma vez que o compilador pode usar as constantes fornecidas diretamente.

\subsection{Argumento final}
\label{macro-rest-argument}

Se o argumento final de uma macro é do tipo \type{Array<Expr>}, a macro aceita um número arbitrário de argumentos extras que estão disponíveis de dentro do array:

import haxe.macro.Expr;

\haxe{assets/MacroArgumentsRest.hx}



\section{Consolidação (Reitification)}
\label{macro-reification}

\extratranslation{
(N. do T.: Reitification pode ser traduzido eventualmente como reitificação, realização, concretização: significando tornar real algo que é abstrato)}

O compilador do Haxe permite a consolidação  \emph{reification} de expressões,  tipos e classes para simplificar o trabalho com macros. A sintaxe para consolidação é \expr{macro expr}, onde \expr{expr} é qualquer expressão valida em Haxe.

\subsection{Consolidação de expressões}
\label{macro-reification-expression}

A consolidação de expressões é usada para criar instâncias de \type{haxe.macro.Expr} de uma forma conveniente. O compilador do Haxe aceita a sintaxe usual do Haxe e a traduz em um objeto expressão. Ele suporta diversos mecânismos de escape, todos disparados pelo caractere \expr{\$}

\begin{description}
   	\item[\expr{\$\{\}} and \expr{\$e\{\}}:] \type{Expr -> Expr} Isso pode ser usado para compor expressões. A expressão delimitada por \expr{\{ \}} é executada, com seu valor sendo utilizado em seu lugar.
    \item[\expr{\$a\{\}}:] \type{Expr -> Array<Expr>} Se usada em um local onde um \type{Array<Expr>} é esperado (por exemplo. chamada de argumentos ou elementos de bloco)  \expr{\$a\{\}} trata seu valor como se fosse aquele array. De outra forma ela gera uma declaração de array.
    \item[\expr{\$b\{\}}:] \type{Array<Expr> -> Expr}  gera um bloco de expressões do array de expressões dado.
    \item[\expr{\$i\{\}}:] \type{String -> Expr}Gera um identificador do string dado.
	\item[\expr{\$p\{\}}:] \type{Array<String> -> Expr} Gera um campo expressão do array de strings dado.
    \item[\expr{\$v\{\}}:] \type{Dynamic -> Expr} Gera uma expressão dependente do tipo de seu argumento. Isso só funciona garantidamente para \tref{tipos básicos}{types-basic-types} e \tref{instâncias de enums}{types-enum-instance}.
\end{description}

Adicionalmente o \tref{metadado}{lf-metadata} \expr{@:pos(p)} pode ser usado para mapear a posição da expressão anotada para \expr{p} ao invés do local onde é consolidada. 

Esse tipo de consolidação só funciona em lugares onde a estrutura interna espera uma expressão. Isso exclui objeto.\expr{object.\$\{nomeDeCampo}\}, mas funciona para \expr{objeto.\$nomeDeCampo}. Isso é verdade em todos os lugares onde a estrutura interna espera um string:

\begin{itemize}
    \item acesso ao campo \expr{objeto.\$nome}
    \item nome da variável \expr{var \$nome = 1};
\end{itemize}
\since{3.1.0}
\begin{itemize}
	\item nome de campo  \expr{\{ \$nome: 1\} }
    \item nome de função \expr{function \$nome() \{ \}}
    \item nome de variável no \expr{try e() catch(\$nome:Dynamic) \{\}}
\end{itemize}


\subsection{Consolidação (reificação) de tipos}
\label{macro-reification-type}

A consolidação de tipos é usada para criar instâncias de \type{haxe.macro.Expr.ComplexType} de uma forma conveniente. Ela é identificada por \expr{macro : Type}, onde Type pode ser qualquer expressão de caminho de tipo válido. Isso é similar a indicação de tipo explícito no código normal, por exemplo, para variáveis na forma \expre{var x:Type}.

Cada constructor de \type{ComplexType} tem uma sintaxe distinta:

\begin{description}
	\item[\expr{TPath}:] \expr{macro : pack.Type}
	\item[\expr{TFunction}:] \expr{macro : Arg1 -> Arg2 -> Return}
	\item[\expr{TAnonymous}:] \expr{macro : \{ field: Type \}}
	\item[\expr{TParent}:] \expr{macro : (Type)}
	\item[\expr{TExtend}:] \expr{macro : \{> Type, field: Type \}}
	\item[\expr{TOptional}:] \expr{macro : ?Type}
\end{description}

\subsection{Consolidação de classes}
\label{macro-reification-class}

Também é possível usar a consolidação para obger uma instância de \type{haxe.macro.Expr.TypeDefinition}. Isso é indicado pela sintaxe \expr{macro class}, como mostrado aqui:

\haxe{assets/ClassReification.hx}

A instância de \type{TypeDefinition} gerada é tipicamente passada para \expr{haxe.macro.Context.defineType} de forma a adicionar um novo tipo ao contexto de chamadas (e não ao próprio contexto da macro)
Esse tipo de consolidação também pode ser útil para obter instâncias de \expr{haxe.macro.Expr.Field}, que estão disponíveis a partir do array \expr{fields} do \type{TypeDefinition} gerado.

\section{Ferramentas}
\label{macro-tools}

A Biblioteca Padrão do Haxe vem com um conjunto de "classes-ferramentas" para simplificar o trabalho com macros. Essas classes funcionam bem como \tref{extensões estáticas}{lf-static-extension} e podem ser trazidas para o contexto ou individualmente ou como um todo através de \expr{using haxe.macro.Tools}. Essas classes são:

\begin{description}
    \item[\type{ComplexTypeTools}:] Permitem escrever instâncias de \type{ComplexType} em um formato legível para humanos. Também permitem determinar o \type{Type} correspondente a um \type{ComplexType}.
    \item[\type{ExprTools}:] Permitem a escrita de instâncias de \expr{Expr} em um formato legível para humanos. Também permitem expressões de iteração e mapeamento.
	\item[\type{TypeTools}:] Oferecem operações úteis sobre strings e expressçoes de strings no contexto de macros.
    \item[\type{TypeTools}:] Permitem a escrita de instâncias de \type{Type} em formato legível para humanos. Também oferecem diversas operações sobre tipos, como \tref{unificação}{type-system-unification} ou conseguir o \type{ComplexType} correspondente.
\end{description}

\trivia{A biblioteca tinkerbell e porque Tools.hx funciona}{Aprendemos sobre as extensões estáticas que usar um \emph{módulo} implica em trazer todos os seus tipos para a o contexto de extensão estática. Disso resulta que, tal tipo pode bem ser um \tref{typedef}{type-system-typedef} para outro tipo. O compilador então considera essa parte tipificada de um módulo e estende a extensão estática concordantemente.

Esse "truque" foi usado na biblioteca \emph{tinkerbell}\footnote{https://github.com/back2dos/tinkerbell} de Juraj Kirchheim exatamente com essa intenção. Tinkerbell ofereceu muitas ferramentas de macro bem antes de leas serem colocadas no Compilador do Haxe e na Biblioteca Padrão do Haxe. Ela permanece uma biblioteca primária para ferramentas de macro adicionais e oferece outras funcionalidades igualmente úteis.}

\section{Montagem de tipos}
\label{macro-type-building}

Macros de montagem de tipo (Type Building) são diferentes de macros de expressões em diversas formas:
\begin{itemize}
    \item Elas não retornam expressões, mas um array de campos de classe. O seu tipo de retorno deve ser explicitamente definido para \type{Array<Haxe.macro.Expr.Field>}.
    \item Seus \tref{contextos}{macro-context} não tem métodos locais nem variáveis locais.
    \item Seus contextos tem campos de montagem (build), disponíveis apartir de \expr{haxe.macro.getBuildFields()}.
    \item Elas não são chamadas diretamente, mas são argumentos para um m\tref{metadado}{lf-metadata} \expr{@:build} ou \expr{@:autobuild} em uma declaração de \tref{classe}{types-class-instance} ou \tref{enum}{types-enum-instance}.

        O exemplo seguinte demonstra a montagem de tipo. Observe que ele é divido em dois arquivos por uma razão: Se um módulo contém uma função de \expr{macro}, ele tem que ser igualmente tipado dentro do contexto de macro. Isso  é frequentemente um problema para macros de montagem de tipos porque o tipo a ser montado só poderia ser carregado em seu estado incompleto, antes que a macro de montagem tenha sido executada. Nós recomendamos a definição de macros de montagem de tipos dentro de seu próprio módulo.

\haxe{assets/TypeBuildingMacro.hx}
\haxe{assets/TypeBuilding.hx}

O método \expr{build} de \type{TypeBuildingMacro} executa três passos:

\begin{enumerate}
    \item Obtém os campos de montagem usando \expr{Context.getBuildFields()}.
    \item Declara um novo campo \type{haxe.macro.expr.Field} usando o argumento de macro \expr{funcName} como nome do campo. Esse campo é uma \tref{variável}{class-field-variable} \type{String} com um valor padrão \expr{"my default"} (do campo \expr{kind}) e é público e estático (do campo \expr{acess}).
    \item Adiciona o novo campo ao array de montagem de campos e o retorna.
\end{enumerate}

Essa macro é o argumento do metadado \expr{@:build} da classe \type{Main}. Tão logo esse tipo seja requerido, o compilador faz o seguinte:

\begin{enumerate}
    \item Analisa a sintaxe do arquivo com o módulo, incluindo os campos de classe.
    \item Define o tipo, incluindo sua relação com outros tipos através de \tref{herança}{types-class-inheritance} e \tref{interfaces}{types-interfaces}.
    \item Executa a macro de montagem de tipo de acordo com o metadado \expr{@:build}.
    \item Continua  a tipificação de classes normalmente com os campos retornados pela macro de montagem de tipos.
\end{enumerate}

Isso permite adicionar e modificar campos de classe a vontade em uma macro de montagem de tipos. Em nosso exemplo a macro é chamada com um argumento \expr{"myFunc"}, fazendo \expr{Main.myFunc} um acesso de campo válido.

Se uma macro de montagem de tipo não modificasse nada, a macro pode retornar \expr{null}. Isso indica ao compilador que nenhuma mudança é pretendida e é preferível o retornar \expr{Context.getBuildFields()}.

\section[Montagem de enum}
\label{macro-enum-building}

A montagem de \tref{enums}{types-enum-instance} é análoga a montagem de classes com um mapeamento simples:

\begin{itemize}
    \item Constructors de enum sem argumentos são campos-variáveis \item{FVar}.
    \item Constructors de enum com argumentos são campos-métodos FFun
\end{description}

\haxe{assets/EnumBuildingMacro.hx}
\haxe{assets/EnumBuilding.hx}

import haxe.macro.Context;
import haxe.macro.Expr;

Porque o enum \type{E} está marcado com um metadado \expr{:build}, a macro chamada monta dois constructors \expr{A} e \expr{B} ``nele''. O primeiro é adicionado com o kind sendo \expr{FVAr (null, null)} implicando que é um constructor sem argumentos. Para o último, usamos \tref{reification}{macro-reification-expression} para obter uma  instância de \type{haxe.macro.Expr.Function} com um único argumento \type{Int}.

O método \expr{main} prova a estrutura do nosso enum gerado testando sua \tref{correspondência}{lf-pattern-matching}. Nós podemos ver que o tipo gerado é equivalente a isso

\begin{lstlisting}
enum E {
    A;
    B(value:Int);
}
\end{lstlisting}



\subsection{@:autobuild}
\label{macro-auto-build}

Se uma classe tem o metadado \expr{:autobuild}, o compilador gera o metadado \expr{:build} em todas as classes que a estendem. Se uma interface tem o metadado \expr{:autobuild}, o compilador gera o metadado \expr{:build} em todas as classes que a implementam ou a estendem. Observe que \expr{:autobuild} não implica em \expr{:build} na própria classe/interface.

\haxe{assets/AutoBuildingMacro.hx}
\haxe{assets/AutoBuilding.hx}

Isso gera saída durante a compilação:

\begin{lstlisting}
AutoBuildingMacro.hx:6:
  fromInterface: TInst(I2,[])
AutoBuildingMacro.hx:6:
  fromInterface: TInst(Main,[])
AutoBuildingMacro.hx:11:
  fromBaseClass: TInst(Main,[])
\end{lstlisting}

É importante manter em mente que a ordem da execução dessas macros não é definida, o que é detalhado em \Fullref{macro-limitations-build-order}.



\subsection{@:genericBuild}
\label{macro-generic-build}
\since{3.1.0}

\tref{Macros de build}{macro-type-building} normais são rodadas por tipos e já são bastante poderosas. Em alguns casos é útil rodar uma macro de build por por \emph{utilização} de tipo, ao invés disso, i.e. em toda a ocasião que realmente aparece no código. Entre outras coisas, isso permite acessar os parâmetros de tipo concretos que aparecem no código.

\expr{@:genericBuild} é usado justamente como \expr{@:build} ao adicioná-lo a um tipo com o argumento sendo uma chamada de macro:

\haxe{assets/GenericBuild1.hx}

Durante a execução esse exemplo gera a saída \ic{TAbstract(Int,[])} e \ic{TInst(String,[])}, indicando que está de fato ciente dos parâmetos de tipo concretos de \type{MyType}. A lógica de macro poderia usar esta informação para gerar um tipo personalizado (usando \expr{haxe.macro.Context.defineType}) ou fazer referência a um existente. Por brevidade, nós retornamos \expr{null} aqui, o que solicita ao compilador para \tref{inferir}{type-system-type-inference} o tipo.

Em Haxe 3.1 o tipo de retorno de uma macro \expr{@:genericBuild} tem que ser um \type{haxe.macro.Type}. Haxe 3.2 permite (e prefere) retornar um \type{haxe.macro.ComplexType} ao invés disso, que é a representação sintática de um tipo. É mais facil de trabalhar com isso em muitos casos porque tipos podem ser simplesmente referenciados por seus caminhos.

\paragraph{Const type parameter}

O Haxe permite a passagens de \tref{expressões constantes}{expression-constants} como um parâmetro de tipo se o nome parâmetro de tipo é \expr{Const}. Isso pode ser utilizado no contexto de macros \expr{@:genericBuild} para passar informação da sintaxe diretamente para a macro:

\haxe{assets/GenericBuild2.hx}

Aqui a lógica de macro poderia carregar um arquivo e usar o seu conteúdo para gerar um tipo personalizado.


\section{Limitações}
\label{macro-limitations}
\state{NoContent}

\subsection{Macro dentro de macro}
\label{macro-limitations-macro-in-macro}

\subsection{Extensão estática}
\label{macro-limitations-static-extension}

Os conceitos de \tref{extensão estática}{lf-static-extensions} e macros são um tanto conflitantes: enquanto o primeiro exige um tipo conhecido de forma a determinar as funções utilizadas, macros são executadas antes da tipagem sobre a sintaxe pura. Não é, portanto, surpreendente que a combinação dessas duas funcionalidades pode levar a pontos de argumentação. O Haxe 3.0 tentaria converter expressões tipadas de volta a expressões sintáticas, o que não é sempre possível e pode perder informações importantes. Nós recomendamos o uso disso com cautela.

\since{3.1.0}

A combinação de extensões estáticas e macros foi retrabalhada para a versão 3.1.0. O compilador do Haxe nem mesmo tenta encontrar a expressão original para o argumento da macro e ao invés disso passa uma expressão especial \expr{@:this this}. Ainda que a estrutura da expressão não traga nenhuma informação, a expressão ainda pode ser tipada corretamente:


\haxe{assets/MacroStaticExtension.hx}



\subsection{Ordem de montagem}
\label{macro-limitations-build-order}

A ordem de montagem dos tipos não é especificada e isso se estende a ordem de execução das  \tref{macros de montagem}{macro-type-building}. Ainda que certas regras possam ser determinadas, nós recomendamos efusivamente que não se fie na ordem de execução das macros de montagem. Se a construção de tipos exige múltiplas passadas, isso não deve ser refletir diretamente no código da macro. De forma a evitar multiplas execuções de macros de montagem sobre o mesmo tipo, o estado pode ser armazendado em variáveis estáticas ou adicionados como \tref{metadados}{lf-metadata} ao tipo em questão:

\haxe{assets/MacroBuildOrder.hx}

Com ambas interfaces \type{I1} e \type{I2} tendo o metadado \expr{:autobuild}, a macro de montagem é executada duas vezes para a classe \type{C}. Previnimos o processo duplicado adicionando um metadado personalizado \expr{:processed} á classe, o que pode ser verificado durante a segunda execução da macro.


\subsection{Parâmetros de tipo}
\label{macro-limitations-type-parameters}


\section{Macros de inicialização}
\label{macro-initialization}

Macros de inicialização são chamadas da linha de comando usando \expr{--macro callExpr(args)}. Isso registra uma uma retrochamada (callback) que o compilador executal depois de criar seu contexto, mas antes de tipificar o que foi passado como argumento para \expr{-main}. Isso, então, permite a configuração do compilador de algumas maneiras.

Se o argumento para \expr{--macro} é uma chamada a um simples identificador, aquele identificador é procurado na classe {haxe.macro.Compiler} que é parte da Biblioteca Padrão do Haxe. A classe vem com diversas macros de incicialização que são detalhadas em sua \href{http://api.haxe.org//haxe/macro/Compiler.html}{API}.

Como um exemplo, a macro \expr{include} permite a inclusão de um pacote inteiro para compilação, recursivamente se necessário. O argumento de linha de comando para isso seria \expr{--macro include ('some.pack', true)}.

É claro que também é possível definir macros personalizadas de inicialização para executar diversas tarefas antes da compilação em si. Uma macro como essa seria chamada via \expr{--macro some.Class.theMacro(args)}. Um caso possível, uma vez que todas as macros compartilham o mesmo \tref{contexto}{macro-context}, uma macro de inicialização poderia atribuir o valor de um campo estático para que outras macros utilizem como configuração.


\part{Standard Library}
% % commit original em 23/mar/2015: 771e7438b9fda344cc3dbb23c9f8ac980c5d44e4
\chapter{Biblioteca padrão}
\label{std}

\translationextra{N.do T.: Texto da Introdução ao Haxe e não do Manual}

A Biblioteca Padrão do Haxe fornece ferramentas para uso geral sem intenção de ser uma coleção exaustiva de estruturas de dados e algoritmos. Uma distribuição de Haxe vem com um diretório \expr{std} contendo a Biblioteca Padrão do Haxe. Seu conteúdo pode ser categorizado como:

\begin{itemize}
    \item [Uso geral:] O diretório std em si contém poucas classes de primeiro nível como Array, Map ou String que podem ser usadas em todos os targets. O subdiretório haxe oferece estruturas de dados adicionais APIs de entrada e saída (input e output, io) e muitas outras ferramentas.

    \item [Sistema:] o subdiretório sys contém APIs relacionadas a sistemas de arquivo e bancos de dados. Adicionalmente as classes de topo permitem várias interações com o sistema operacional. Elas também podem ser acessadas quando se compila para um target da categoria sys-category(C++,C#,Java,Neko,PHP).

    \item [Específicas de targets:] Cada target do Haxe tem um subdiretório distinto contendo APIs específicas àquele target; Essas só podem ser acessadas quando se compila para o dado target.
\end{itemize}

API de uso geral:

Nível superior:

Array: Coleção tipada que define diversas operações de acordo com as especificações ECMA

Datas, Ferramentas para datas: Operações relativas a datas e marcação de horário (timestamps)

EReg: Expressões Regulares

Lambda: Operações sobre iteráveis

Map: estrutura de dados de mapeamento chave-para-valor

Math: Funções matemáticas de acordo com as especifiações ECMA

Reflect: reflexão relacionada a campos

Std: Verificação de tipos em tempo de execução, análise sintática de números (parsing), conversão para Int e String

String: operações básicas sobre Strings

StringBuf: Otimizada para montagem de Strings

StringTolls: Varias extensões a Strings

Type: Reflexão relacionada a Type

XML: XML entre plataformas

O pacote haxe:

haxe.Http: Requisições HTTP

haxe.Json: Codificando e decodificando JSON

haxe.Resource: Trabalha com recursos do Haxe

haxe.Serializer: Serializa objetos arbitrários como String

haxe.Template: Um sistema simples de templates (modelos, gabaritos)

haxe.Timer: Execução repetida/adiada; cronômetro

haxe.Unserializer: complemento do haxe.Serializer

haxe.Utf8: Strings UTF8 entre plataformas

haxe.crypto: vários algoritmos de encriptação

haxe.macro: Tipos para trabalhar com macros de Haxe

haxe.remoting: para criar aplicações remotas entre varios tipos de cliente e servidores

haxe.rtti: Informações de tipo em tempo de execução

haxe.unit: Ambiente de testes de unidades básicas

haxe.web: Mapas de URLs para operações

haxe.xml: ferramentas complementares de XML

haxe.zip: suporte ao formato zip

O pacote haxe.ds.package:

haxe.ds.ArraySort: Ordenamento de array multiplataforma estável

haxe.ds.BalancedTree: estrutura de dados de árvore equilibrada

haxe.ds.EnumValueMap: Tipo Map com suporte para chaves de valores de enum

haxe.ds.GenericStack: Estrutura de dados de stack que é otimizada em targets estáticos

haxe.ds.IntMap: Tipo Map com suporte para chaves Int

haxe.ds.ObjectMap: Tipo Map com suporte a chaves object

haxe.ds.StringMap: Tipo Map com suporte a chaves String

haxe.ds.Vector: estruturas de dado de tamanho fixo

O pacote haxe.io 

haxe.io.Bytes: Operações em bytes sobre representações nativas

haxe.io.BytesBuffer: Otimizada para montagem do tipo Bytes (haxe.io.Bytes)

haxe.io.Path: Operações sobre strings de caminhos

API de Sistema:

Disponíve em C++, C#, Java, Neko e PHP.

Sys: Executa comandos nativos, interage com stdin, stdout e stderr; várias outras operações nativas.

sys.FileSystem: Lê e modifica diretórios; obtém informações sobre arquivos e diretórios

sys.db: APIs para trabalhar com bancos de dados MySQL e SQLite

sys.io.File: Lê e escreve conteúdo de arquivos; copia arquivos

sys.io.Process: usa processos nativos

APIs específicas de targets:

- cpp:
 --cpp.Lib: Interações de baixo nível com o target cpp
 --cpp.net: Ferramentas para interagur com redes e servidores em execução
 --cpp.vm: API de threads, debugger, profiler, etc.
 --cpp.zip: API para trabalhar com compressão zip

- cs: API para o target C#

- flash:
 -- flash: Externalizações para a API de Flash
 -- flasg.Lib: Interações básicas com a plataforma Flash
 -- flash.Memory: Externalizações para a API Memory do Flash
 -- flash.Vector: Externalização para Vectors de Flash

- flash8:
 -- flash8: Externalizações para API do Flash 8

- java: API para o target Java

- js:
 --js.Browser: Atalhos para funções comuns de browsers
 --js.Cookie: Assistentes para interagir com cookies HTTP no browser
 --js.JQuery: Classes externalizadas e assistentes para JQuery
 --js.Lib: Atalhos para alert(), eval() e debugger
 --js.html: Externalizações para interagir com o DOM do browser

- neko:
 -- neko.Lib: Interações de baixo nível com a plataforma neko
 -- neko.Web: Funciona com requisições e respostas HTTP
 -- neko.net: Ferramentas para trabalhar com redes e servidores em execução -- neko.vm: API para aplicações multi-threaded
 -- neko.zip: API para trabalar com compressão zip

- php:
 -- php.Lib: Interação de baixo nível com a plataforma PHP
 -- php.Session: Funciona com seções nativas de PHP
 -- php.Web:Funciona com requisições e resposta HTTP
 -- php.db.PDO: driver adicional de PDO para interação com bancos de dados

 
\translationextra{Fim do Texto da Introdução ao Haxe e não do Manual}


class Main {
    static public function main() { }
}


\section{String}
\label{std-String}


\define[Type]{String}{Um String é uma sequência de caracteres.}

\section{Estruturas de dados}
\label{std-ds}
\state{NoContent}

\subsection{Array}
\label{std-Array}

Um \type{Array} é uma coleção de elementos. Ele tem um \tref{parâmetro de tipo}{type-system-type-parameters}  que corresponde aos tipos desses elementos. Arrays podem ser criados de três formas:
\begin{enumerate}
    \item Usando seu constructor: \expr{new Array()}
    \item Usando a \tref{sintaxe de declaração de array}{expression-array-declaration}: \expr{[1, 2, 3]}
    \item Usando o \tref{preenchimento de array}{lf-array-comprehension}: \expr{[for (i in 0...10) if (i \% 2 == 0) i]}
\end{enumerate}

Arrays vem com uma \href{http://api.haxe.org/Array.html}{API} para atender a maioria dos casos de uso (use-cases). Adicionalmente eles permitem \tref{acesso a escrita e leitura dos arrays}{expression-array-access}:

\haxe{assets/ArrayAccess.hx}

Uma vez que o acesso a arrays in Haxe é ilimitado, isto é, se garante que le não lance uma mensagem de excessão, isso exige maior discussão:
\begin{itemize}
    \item Se um acesso de leitura é feito em um índice não existente, um valor dependente do target é retornado.
    \item Se um acesso de escrita é feiuto com um índice positivo que é além do limite, null (ou o valor padrão (ver seção) de tipos básicos (ver seção) em targets estáticos (ver seção)) é inserido em todas as posições entre o último índice definido e o recém-escrito.
    \item Se o acesso é feito com um índice negativo, o resultado é inespecífico.
\end{itemize}

Arrays definem um \tref{iterador}{lf-iterators} sobre seus elementos. Essa iteração é tipicamente otimizada pelo compilador para uso com \tref{loop \expr{while}}{expression-while}  para o índice do array:

\haxe{assets/ArrayIterator.hx}

O Haxe gera a seguinte saída em \target{Javascript} otimizado:

\begin{lstlisting}
Main.main = function() {
    var scores = [110,170,35];
    var sum = 0;
    var _g = 0;
    while(_g < scores.length) {
        var score = scores[_g];
        ++_g;
        sum += score;
    }
    console.log(sum);
};
\end{lstlisting}

O Haxe não permite arrays com tipos mistos ao menos que o tipo de parâmetro seja forçado como \tref{\type{Dynamic}}{types-dynamic}:

\begin{lstlisting}
class Main {
    static public function main()
        // Erro de compilação: Arrays de tipo misto só
        // são permitidos se o tipo é forçado  para
        // Array<Dynamic>
        //var myArray = [10, "Bob",false];

        // Array<Dynamic> with mixed types
        var myExplicitArray:Array<Dynamic> =
        [10, "Sally", true];
    }
}
/end{lstlisting}

\trivia{Arrays dinâmicos}{Em Haxe 2, declaração de arrays de tipo mistos eram permitidas. Em Haxe 3, arrays só podem ter tipos mistos se forem explicitamente declarados como \expr{Array<Dynamic>}.}


\subsection{Vetor(haxe.ds.Vector)}
\label{std-vector}

Um \type{Vetor} é uma \emph{coleção} otimizada de elementos de tamanho fixo. Parecido com \tref{Array}{std-Array}, tem um \tref{parâmetro de tipo}{type-system-type-parameters} e todos os elementos do vetor devem ser do tipo especificado; ele pode ser \emph{iterado} com o uso de um \tref{loop for}{expression-for}  e acessado utilizando a \tref{sintaxe de array}{types-abstract-array-access}. Entretanto, diferente dos tipos \type{Array} e \type{List}, o comprimento de um vetor é especificado na sua criação e não pode ser mudado depois.

\haxe{assets/Vector.hx}

\type{haxe.ds.Vector} é implementado como um tipo abstrato (\ref{types-abstract}) sobre a implementação nativa do array para um dado target e pode ser mais rápido para coleções de tamanho fixo, porque a memória para armazenamento de seus elementos é pré-alocada.

\subsection{Lista (haxe.ds.list)}
\label{std-List}

Uma lista (\type{List}) é uma \emph{coleção} para armazenamento de elementos. Na superfície uma lista é similar a um \Fullref{std-Array}. Entretando, a implementação subjacente é muito diferente, isso resulta em diversas diferenças de funcionamento:

\begin{enumerate}
    \item Uma lista não pode ser indexada com o uso de colchetes, i.e. [0].
    \item Uma lista não pode ser inicializada.
    \item Não há preenchimento de listas
    \item Uma lista pode modificar/adicionar/subtrair elementos livremente enquanto é iterada.
\end{enumerate}

Veja a \href{http://api.haxe.org/List.html}{List API} para detalhes sobre os métodos de listas. Um simples exemplo para trabalhar com listas:

\haxe{assets/ListExample.hx}

\subsection{Pilha/Stack Genérico (GenericStack)}
\label{std-GenericStack}

Um \type{GenericStack}, como \type{Array} e \type{List} é um container para armazendagem de elementos. Ele tem um  \tref{parâmetro de tipo}{type-system-type-parameters} e todos os elementos do stack devem ser do tipo especificado. Veja a \href{http://api.haxe.org/haxe/ds/GenericStack.html}{API GenericStack}  para detalhes sobre seus métodos. Aqui um pequeno exemplo de um programa para a inicialização e uso de um \type{GenericStack}.

\haxe{assets/GenericStackExample.hx}

\trivia{Lista Rápida}{Em Haxe 2, a classe GenericStack era conhecida com FastList. Uma vez que seu comportamento é bem parecdio com um stack típico, o nome foi mudado para o Haxe 3}

O \emph{Generic} em \type{GenericStack} é literal. É atribuído com o metadado \expr{:generic}. Dependendo do target, isso pode levar a melhores desempenhos em targets estáticos. Veja \Fullref{type-system-generic} para mais detalhes.

\subsection{Map}
\label{std-Map}

Um\type{Map} é um contêiner composto de pares \emph{chave},\emph{valor}. Um \type{Map} também é normalmente tratado por array associativo, dicionário ou tabela de símbolos. O código seguinte dá um breve exemplo do funcionamento de maps:

\haxe{ptassets/MapExample.hx}

Veja a \href{http://api.haxe.org/Map.html}{API de Maps}  para detalhes sobre seus métodos.

Debaixo do capô, um \type{Map} é um tipo abstrato (ver seção). Em tempo de compilação ele é convertido de uma diversos tipos especializados dependendo do tipo da \emph{chave}:

\begin{itemize}
    \item \type{String}: \type{haxe.ds.StringMap}
    \item \type{Int}: \type{haxe.ds.IntMap}
    \item \type{EnumValue}: \type{ haxe.ds.EnumValueMap}
    \item \type{\{\}}: \type{haxe.ds.ObjectMap}
\end{itemize}

O tipo \type{Map} não existe em tempo de execução e foi substituído por um dos objetos acima.

Map defince o \tref{acesso como array}{types-abstract-array-access} usando o tipo de sua chave.

\subsection{Option}
\label{std-Option}

Um option é um \tref{enum}{types-enum-instance} na Biblioteca Padrão do Haxe; é definido assim:

\begin{lstlisting}
enum Option<T> {
	Some(v:T);
	None;
}
\end{lstlisting}

Ele pode ser usado em várias situações, como para comunicar se um método tem ou não um retorno válido e, caso tenha, qual valor retornou:

\haxe{assets/OptionUsage.hx}

\section{Expressões Regulares}
\label{std-regex}

O Haxe tem suporte interno a \emph{expressões regulares} \footnote{http://pt.wikipedia.org/wiki/Express\%C3\%A3o_regular}. Elas podem ser usadas para verificar o formato de um string, transformar um string ou extrair dados regulares de um dado texto.

    O Haxe tem uma sintaxe especial para a criação de expressões regulares. Nós podemos criar um objeto expressão regular digitando-a entre a {combinação de caracteres \expr{\textasctilde/} e uma barra simples \expr{/}:

\begin{lstlisting}
var r = ~/haxe/i;
\end{lstlisting}

Alternativamente, podemos criar uma expressão regular com a sintaxe normal:

\begin{lstlisting}
var r = new EReg("haxe", "i");
\end{lstlisting}

O primeiro argumento é uma string com um padrão de expressão regular, o segundo é um string com um sinalizador (flag, ver abaixo).

Podemos usar padrões usuais de expressões regulares como:
\begin{itemize}
    \item \expr{.} qualquer caractere
    \item \expr{*} repetir zero ou mais vezes
    \item \expr{+} repetir uma ou mais vezes
    \item \expr{?} opcional zero-ou-um
    \item \expr{[A-Z0-9]} intervalos de caracteres
  	\item \expr{[\textasciicircum\textbackslash r\textbackslash n\textbackslash t]} caractere fora do intervalo
    \item \expr{(...)} parentes para localizar grupos de caracteres
	\item \expr{\textasciicircum} começo do string (começo de uma linha em modo de localização multilinhas)
	\item \expr{\$}fim do string (fim da linha no modo de localização multilinhas)
    \item \expr{|}declaração "OU"
\end{itemize}

\translationextra{N.do T.:\expr{\textasciicircum= negação}: \textasciicircum c = não c, \textasciicircum (x-z) = diferente de x, y ou z}
%acho que \r caractere de retorno de carro (carriage return)
%\t caractere de tabulação (tab)
%\n caractere de nova linha (que depende do OS?) 

Por exemplo, a expressão regular seguinte localiza endereços de e-mail válidos:
\begin{lstlisting}
~/[A-Z0-9._\%-]+@[A-Z0-9.-]+\.[A-Z][A-Z][A-Z]?/i;
\end{lstlisting}

Por favor, observe que o \expr{i} ao final da expressão regular é um sinalizador (flag) que ativa a indiferença entre maiúsculas e minúsculas na localização.

Os sinalizadores possíveis são os seguintes:
\begin{itemize}
    \item \expr{i}indiferença a maiúsculas e minúsculas
    \item \expr{g} substituição global ou divisão, veja abaixo
    \item \expr{m} modo de localização multilinha,\expr{\textasciicircum} e \expr{\$}  representam o começo e o final de uma linha
    \item \expr{s} o ponto \expr{.} também ``casa'' com novas linhas \emph{ (apenas nos targets Neko, C++, PHP e Java)}
    \item \expr{u} usa localização com UTF-8 \emph{(apenas targets Neko e C++)}
\end{itemize}

\subsection{Casamento de padrões (Matching)}
\label{std-regex-match}

Provavelmente um dos usos mais comuns para expressões regulares é a verificação de encaixe de um string a um determinado padrão específico. O método \expr{match} de um objeto expressão regular pode ser usado para fazer isso:

\haxe{assets/ERegMatch.hx}

\subsection{Grupos}
\label{std-regex-groups}

Informações específicas podem ser extraídas de um string coincidente com padrão através do uso de \emph{grupos} (\emph{groups}). Se \expr{match()} retornar true (verdadeiro), podemos pegar grupos usando o método \expr{matched(X)}, onde X é o número do grupo definido pelo padrão da expressão regular:

\haxe{assets/ERegGroups.hx}

Observe que os números dos grupos começam em 1 e que \expr{r.matched(0)} sempre retorna todo o substring localizado.

\expr{r.matchedPos()} retornará a posição desse substring no string original:

\haxe{assets/ERegMatchPos.hx}

Adicionalmente, \expr{ r.matchedLeft()} e \expr{r.matchedRight()} podem ser usados para pegar substrings à esquerda e a direita do substring localizado:

\haxe{assets/ERegMatchLeftRight.hx}

\subsection{Replace}
\label{std-regex-replace}

Uma expressão regular também pode ser usada para substituir uma parte do string:

\haxe{assets/ERegReplace.hx}

Podemos usar \expr{\$X} para reutilizar um grupo localizado na substituição:

\haxe{assets/ERegReplaceGroups.hx}

\subsection{Split(Partilha)}
\label{std-regex-split}

Uma expressão regular pode também ser usada para partir um string em diversos substrings:

\haxe{assets/ERegSplit.hx}

\subsection{Map}
\label{std-regex-map}

O método \expr{map} de um objeto expressão regular pode ser usado para substituir substrings usando uma função personalizada. Essa função toma um objeto de expressão regular como primeiro argumento de forma que possamos usá-lo para tirar informações adicionais da correspondência sendo feita. Por exemplo

\haxe{assets/ERegMap.hx}

\subsection{Detalhes de Implementação}
\label{std-regex-implementation-details}

Expressões regulares são implementadas:

\begin{itemize}
    \item em JavaScript, a execução oferece a implementação com o objeto RegExp
    \item em Neko e C++, a biblioteca PCRE é utilizada
    \item em Flash, PHP, C# e Java, implementações nativas são usadas
    \item em Flash 6/8, a implementação não está disponível
\end{itemize}

\section{Bibllioteca Matemática(Math)}
\label{std-math}

O Haxe inclui uma biblioteca matemática de ponto flutuante para algumas operações mateméticans comuns. A maioria das funções opera sobre e retorna \type{floats}. Entretanto um \type{Int} pode ser usado onde um \type{Float} é esperado, e o Haxe também converte \type{Int} para \type{Float} durante a maioria das operações numéricas (ver \Fullref{types-numeric-operators} para maiores detalhes)

Aqui estão alguns exemplos de uso da biblioteca matemática. Ver \href{http://api.haxe.org/Math.html}{Math API} para todas as funções disponíveis.

\haxe{assets/MathExample.hx}

\subsection{Números especiais}
\label{std-math-special-numbers}

A biblioteca matemática tem definições para diversos números especiais:

\begin{itemize}
    \item NaN (Not a Number): retornado quando uma operação matematicamente incorreta é executado, e.g.,Math.sqrt(-1)
    \item POSITIVE_INFINITE: divisão de um número positivo por zero
    \item POSITIVE_NEGATIVE: divisão de um número negativo por zero
	\item PI : 3.1415...
\end{itemize}

\subsection{Erros matemáticos}
\label{std-math-mathematical-errors}

Ainda que o neko possa manusear erros matemáticos com fluidez, como a a divisão por zero, isso não é verdade para todos os targets. Dependendo do target, erros matemáticos podem produzir excessões e ultimamente, erros.

\subsection{Matemática com Inteiros}
\label{std-math-integer-math}

Se você está compilando para uma plataforma que pode utilizar operações com inteiros, e.g., divisões de inteiros, você deve envelopar isso no \emph{Std.int()} para melhor desempenho. O compilador do Haxe pode então, otimizar o resultado para operações inteiras. Um exemplo:

\begin{lstlisting}
	var intDivision = Std.int(6.2/4.7);
\end{lstlisting}

\subsection{Extensões}
\label{std-math-extensions}

É comum ver Extensões Estáticas \Fullref{lf-static-extension} usadas com a biblioteca matemática. Este código mostra um exemplo simples:

\haxe{assets/MathStaticExtension.hx}
\haxe{assets/MathExtensionUsage.hx}

\section{Lambda}
\label{std-Lambda}

\define{Lambda}{define-lambda}{Lambda é um conceito de linguagem funcional dentro do Haxe que permite a você aplicar uma função a uma lista ou a \tref{iteradores}{lf-iterators}. A classe Lambda é uma coleção de métodos funcionais de forma a usar estilo de programação funcional com Haxe.}

É idealmente usado com \expr{using Lambda} (ver \tref{Static Extension}{lf-static-extension}) e então atua como uma extensão aos tipos \type{Iterable}.

Em plataformas estáticas, trabalhando com a estrutura \type{Iterable} pode ser mais lento do que desempenhando as operações diretamente nos tipos conhecidos, tasi como \type{Array} e \type{List}.

\paragraph{Funções Lambda}

A classe Lambda permite que nós operemos sobre todo um \type{Iterable} de uma vez.
Isso é geralmente preferível a rotinas de loops uma vez que é menos inclinado a erros e mais fácil de ler.
Para conveniência, as classes \type{Array} e \type{List} contém alguns dos métodos frequentemente usados da classe Lambda.

É de ajuda olhar para um exemplo. A função exists é especificada como:

\begin{lstlisting}
static function exists<A>( it : Iterable<A>, f : A -> Bool ) : Bool
\end{lstlisting}

A maioria das funções Lambda são chamadas de formas similares. O primeiro argumento para todas as funções Lambdas é o iterável (\type{Iterable}) sobre o qual operar. Muitas também tomam uma função como argumento.

\begin{description}
	\item[\expr{Lambda.array}, \expr{Lambda.list}] Converte Iterable para tipo \type{Array} ou \type{List}. Sempre retorna uma nova instância.
	\item[\expr{Lambda.count}] Conta o número de elemento. Se o iterável é um \type{Array} ou \type{List} é mais rápido usar sua propriedade lenght.
    \item[\expr{Lambda.empty}] Determina se o iterável é vazio. Para todos iteráveis é melhor usar essa função, também é mais rápido do que comparar length (ou o resultado de Lambda.count) com zero.
	\item[\expr{Lambda.has}] Determina se o elemento especificado está no Iterável.
	\item[\expr{Lambda.exists}] Determina se um critério é satisfeito por um elemento.
	\item[\expr{Lambda.indexOf}] Acha o índice do elemento especificado.
	\item[\expr{Lambda.find}] Acha o primeiro elemento de uma dada função de busca.
	\item[\expr{Lambda.foreach}] Determina se cada elemento satisfaz um critério.
	\item[\expr{Lambda.iter}] Chama uma função para cada elemento.
	\item[\expr{Lambda.concat}] Funde dois Iteráveis, retornando uma nova lista.
	\item[\expr{Lambda.filter}] Acha os elementos que satisfazer um critério, retornando uma nova lista.
	\item[\expr{Lambda.map}, \expr{Lambda.mapi}] Aplica uma conversão a cada elemento, retornando uma nova list.
    \item[\expr{Lambda.fold}] ``Fold funcional'' também é conhecido como reduz, acumula, comprime ou injeta.
\end{description}

Esse exemplo demonstra o filtro Lambda e map sobre um conjunto de strings:

\begin{lstlisting}
using Lambda;
class Main {
    static function main() {
        var words = ['car', 'boat', 'cat', 'frog'];

		var isThreeLetters = function(word) return word.length == 3;
		var capitalize = function(word) return word.toUpperCase();
		
		// Three letter words and capitalized. 
		trace(words.filter(isThreeLetters).map(capitalize)); // [CAR,CAT]
    }
}
\end{lstlisting} 

Esse exemplo demontra as funções Lambda count, has, foreach and fold sobre um conjunto de inteiros.

\begin{lstlisting}
using Lambda;
class Main {
    static function main() {
        var numbers = [1, 3, 5, 6, 7, 8];
		
		trace(numbers.count()); // 6
		trace(numbers.has(4)); // false
		
        // test if all numbers are greater/smaller than 20
		trace(numbers.foreach(function(v) return v < 20)); // true
        trace(numbers.foreach(function(v) return v > 20)); // false
		
        // sum all the numbers
		var sum = function(num, total) return total += num;
		trace(numbers.fold(sum, 0)); // 30
    }
}
\end{lstlisting} 

\section{Template}
\label{std-template}

Haxe vem com um sistema padrão de template com uma sintaxe fácil de usar que interpretado por uma classe 'peso-leve' chamada \type{haxe.Template}.

Um template é um string ou um arquivo que é usado para produzir qualquer espécie de saída de string dependendo da entrada. Aqui está um pequeno exemplo de template 

\haxe{assets/Template.hx}

O console irá executar trace \ic{Meu nome é Mark, 30 anos}.

\paragraph{Expressões}
Uma expressão pode ser colocada entre \ic{::}, a sintaxe permite as atuais possibilidades:

\begin{description}
	\item[\ic{::nome::}] a variável nome
	\item[\ic{::expr.campo::}] acesso a campo
	\item[\ic{::(expr)::}] a expressão expr é calculada
	\item[\ic{::(e1 op e2)::}] a operação op é aplicada a e1 e a e2
	\item[\ic{::(135)::}] o integer 135. Constantes float não são permitidas.
\end{description}

\paragraph{Condições}
É possível testar condições usando \ic{::if flag1::}. Opcionalmente, a condição deve ser seguida por \ic{::elseif flag2::} ou \ic{::else::}. Encerre a condição com \ic{::end::}.

\begin{lstlisting} 
::if isValid:: valid ::else:: invalid ::end::
\end{lstlisting} 

Operadores podem ser usados, mas não lidam com precedência de operadores. Portanto é exigido cercar cada operação por parenteses \ic{()}. Atualmente os seguintes operadores são permitidos: \ic{+}, \ic{-}, \ic{*}, \ic{/}, \ic{>}, \ic{<},  \ic{>=}, \ic{<=}, \ic{==}, \ic{!=}, \ic{&&} and \ic{||}.

Por exemplo \ic{::((1 + 3) == (2 + 2))::} exibirá true.

\begin{lstlisting} 
::if (points == 10):: Ótimo! ::end::
\end{lstlisting} 

Para comparar com um string use aspas duplas\ic{"} no template.
\begin{lstlisting} 
::if (nome == "Mark"):: Oi Mark ::end::
\end{lstlisting} 

\paragraph{Iterating}
Itere em uma estrutura ao usar \ic{::foreach::}. Termine o loop com \ic{::end::}.
\begin{lstlisting} 

<table>
	<tr>
		<th>Nome</th>
		<th>Idade</th>
	</tr>
	::foreach users::
		<tr>
			<td>::nome::</td>
			<td>::idade::</td>
		</tr>
	::end::
</table>
\end{lstlisting} 

\paragraph{Sub-templates}
Para incluir templates em outros templates passe o string resultante do subtemplate como um parâmetro
.
\begin{lstlisting} 
var users = [{nome:"Mark", idade:30}, {nome:"John", idade:45}];

var userTemplate = new haxe.Template("::foreach users:: ::nome::(::idade::) ::end::");
var userOutput = userTemplate.execute({users: users});

var template = new haxe.Template("Os usuários são ::users::");
var output = template.execute({users: userOutput});
trace(output);
\end{lstlisting} 
O console executará o trace \ic{Os usuários são Mark(30) John(45)}.

\paragraph{Macris de templates}

Para chamar funções personalizadas enquanto partes do template estão sendo reinderizadas, forneça um objeto \expr{macro} como argumento de \expr{Template.execute}. A chave atuará como o nome da variável do template, o valor se referirá a uma função de retorno (call back function) que deve devolver um \type{String}. O primeiro argumento dessa macor é sempre um método \expr{resolve()} seguido pelos argumentos dados. A função resolve pode ser chamada para reter valores do contexto do template. Se \expr{macros} não tiver esse campo, o resultado é não especificado.

O exemplo seguinte passa a si mesmo como contexto de função macro e executa \ic{display} do template.
\haxe{assets/TemplateMacros.hx}
O console executará o trace \ic{Os resultados: Mark correu 3.5 quilometros em 15 minutos}.

\paragraph{Globals}
Use o objeto \expr{Template.globals} para armazenar valores que deveriam ser aplicados ao largo de de todas as instâncias de \type{haxe.Template}. Isso tem menor prioridade que o argumento de contexto de \expr{Template.execute}.

\paragraph{Using resources}

Para separar o conteúdo do código, considera usar o \tref{sistema de embarque de recursos}{cr-resources}. 
Coloque o conteúdo do template e um novo arquivo chamado \ic{sample.mtt}, e adicione \ic{-resource sample.mtt@my_sample} aos argumentos do compilador e retenha o conteúdo usando \expr{haxe.Resource.getString}.

\haxe{assets/TemplateResource.hx}

Quando executar o template system do lado servidor, você pode simplesmente usar \expr{neko.Lib.print} ou \expr{php.Lib.print} ao invés do trace para mostrar o template de HTML template para o usuário.


\section{Reflexão}
\label{std-reflection}

O Haxe suporta reflexão de tipos e campos em tempo de execução. Cuidado especial deve ser tomado aqui porque a representação em tempo de execução geralmente vária de um target para outro. De forma a usar reflexão corretamente é necessário entender que tipos de operações são suportadas e quais não. Dada a natureza dinâmica da reflexão, isso nem sempre pode ser determinado em tempo de compilação.

A API de reflexão consiste em duas classes:

\begin{description}
    \item[Reflect:] Uma API leve que funciona melhor em \tref{estruturas anônimas}{types-anonymous-structure}, com suporte limitado para \tref{classes}{types-class-instance}. 
    \item[Type:] Uma API mais robusta para trabalhar com classes e \tref{enums}{types-enum-instance}.

Os métodos disponíveis são detalhados nas API \href{http://api.haxe.org//Reflect.html}{Reflect} e \href{http://api.haxe.org//Type.html}{Type}.

A reflexão pode ser uma ferramenta poderosa, mas é importante entender porque ela também pode causar problemas. Como um exemplo, diversas funções esperam um argumento \tref{String}{std-String}  e tentam resolvê-lo para um tipo ou campo. Isso é vunerável a erros de digitação:

\haxe{assets/ReflectionTypo.hx}

Entretanto, mesmo que não existam erros de digitação é fácil esbarrar em comportamentos inesperados:

\haxe{assets/ReflectionMissingType.hx}

O problema aqui é que o compilador nunca chega a `` ver'' realmente a tipo  \type{haxe.Template}, então não o compila para a saída. Além disso, mesmo que ele visse o tipo, poderiam aparecer problemas surgindo com a \tref{eliminação de código morto}{cr-dce}, onde tipos e campos que só são utilizados via reflexão sejam eliminados.

Um outro conjunto de problemas deriva do fato que, por concepção, diversas funções de reflexão esperam argumentos do tipo type \tref{Dynamic}{types-dynamic}, implicando que o compilador não pode verificar se os argumentos passados estão corretos. O exemplo seguinte mostra um engano normal quando se trabalha com \expr{callMethod}:

\haxe{assets/ReflectionMissingType.hx}

A chamada comentada seria aceita pelo compilador porque ela atribui o string \expr{"f"} como argumento da função \expr{func}, especificada para ser do tipo \expr{Dynamic}.

Um bom conselho para trabalhar com reflexão é envelopá-la em algumas poucas funções dentro da aplicação ou API que são chamadas por, de outra forma, código com tipagem segura. Um exemplo poderia ser parecido com isso:

\haxe{assets/ReflectionWrap.hx}

Ainda que o método reflective pudesse trabalhar inteiramente com reflexão (e o tipo Dynamic para tanto), seu tipo de retorno é uma estrutura tipada que as chamadas podem usar de uma maneira segura em relação a tipos. 

\section{Serialização}
\label{std-serialization}

Muitos valores de tempo de execução podem ser seriados e deseriados usando as classes \type{haxe.Serializer} e \type{haxe.Unserializer}. Ambas suportam dois usos:

\begin{enumerate}
    \item Criação de uma instância e continuas chamadas do método \expr{serialize}/\expr{unserialize} para manusear múltiplos valores.
    \item Chamada do seu método estático \expr{run} para serializar/deserializar um único valor.
\end{enumerate}

O exemplo seguinte mostra primeiro uso:

\haxe{assets/SerializationExample.hx}
 
O resultado da serialização (aqui armazendado na variável local \expr{s}) é um \tref{String}{std-String} que pode ser jogado para lá e para cá, mesmo remotamente. Seu formato é descrito em \Fullref{std-serialization-format}.

\paragraph{Valores suportados}

    \item \expr{null} 
    \item \type{Bool}, \type{Int} e \type{Float} (incluindo infinitos e \expr{NaN})
    \item \type{String}
    \item \type{Date}
    \item \type{haxe.io.Bytes} (codificados em base64)
	\item \type{haxe.io.Bytes} (encoded as base64)
	\item \tref{\type{Array}}{std-Array} e \tref{\type{List}}{std-List}
	\item \type{haxe.ds.StringMap}, \type{haxe.ds.IntMap} e \type{haxe.ds.ObjectMap}
	\item \tref{estruturas anônimas}{types-anonymous-structure}
	\item Haxe \tref{instâncias de classe}{types-class-instance} de Haxe (não as nativas)
    \item \tref{instâncias de enum}{types-enum-instance}
\end{itemize}

\paragraph{Configuração de serialização}

A serialização pode ser configurada de duas maneiras. Para ambas, uma variável estática pode ser definida para influenciar todas as instâncias de \type{haxe.Serializer} e uma variável membro pode ser definida para influenciar apenas uma instância específica:

\begin{description}
    \item[\expr{USE_CACHE}, \expr{useCache}:]  useCache: Se verdadeira, objetos repetidos são seriado por referência. Isso pode evitar loops infinitos para dados recursivos ao custo de um tempo maior de serialização. Por padrão, esse valor é \expr{false}.
    \item[\expr{USE_ENUM_INDEX}, \expr{useEnumIndex}:]  useEnumIndex:Se verdadeira, constructors de enum são seriados por seu índice em vez de seu nome. Isso pode tornar a serialização de strings mais curta, mas se quebra se constructors de enum são inseridos nos tipos antes da deserialização. Por padrão, esse valor é \expr{false}.
\end{description}

\paragraph{Comportamento de deserialização}

Se o resultado de serialização é armazenado e depois usado para deserialização, cuidado deve ser tomado para manter a compatibilidade quando se trabalha com instâncias de enums e classes. É, então, importante entender exatamente como a deserialização é implementada.

\begin{itemize}
	\item  O tipo tem que estar disponível em tempo de execução aonde a deserialização é feita. Se a \tref{eliminação de código morto}{cr-dce} está ativa, um tipo que só é usado através de serialização pode ser removido.
    \item Cada \type{Unserializer} tem uma variável membro \expr{resolver} que é usada para resolver classes e enums por nome. Na criação do \type{Unserializer}, isso é usado para definir \type{Unserializer.DEFAULT_RESOLVER.} Tanto ela quanto o membro da instância pode ser definidos para  definir um "resolver" personalizado.
    \item Classes são resolvidas por nome, usando \expr{resolver.resolveClass(nome)}. A instância é, então, criada usando \expr{Type.createEmptyInstance}, o que implica que o constructor da classe não é chamado. Finalmente, os campos da instância são definidos de acorodo com o valor serializado
    \item Enums são resolvidos por nome usando \expr{resolver.resolveEnum(nome)}. A instância do enum é, então, criada usando \expr{Type.createEnum}, usando os valores de argumento serializados, se disponíveis. Se os argumentos do constructor foram mudados desde a serialização, o resultado é não especificado.
\end{itemize}

\paragraph{(De)Serialização personalizada}

Se uma classe defini o método membro \expr{hxSerialize}, esse método é chamado pelo serializador, permitindo uma serialização personalizada da classe. Da mesma forma, se uma classe define o método membro \expr{hxUnserialize}, ele é chamado pelo deserializador:

\haxe{assets/SerializationCustom.hx}

Nesse exemplo decidimos que vamos ignorar o valor membro da variável \expr{y} e não serializá-lo. Ao invés disso o padronizamos para \expr{-1} em \expr{hxUnserialize}. Ambos os métodos são anotados com o metadado \expr{:keep} para previnir que a \tref{eliminação de código morto}{cr-dce} os removam, uma vez que elas nunca são propriamente referenciadas no código.

\subsection{Formato de Serialização}
\label{std-serialization-format}

Cada valor suportado é traduzido para um caracter de prefixo distinto, seguido pelos dados necessários.

	\item[\expr{null}:] \expr{n}
	\item[\type{Int}:] \expr{z} para zero, ou \expr{i} seguido do próprio integer (e.g. \expr{i456})
    \item[\type{Float}:]
		\begin{description}
			\item[\expr{NaN}:] \expr{k}
			\item[infinito negativo:] \expr{m}
			\item[infinito positivo:] \expr{p}
			\item[float normal:] \expr{d} seguido da exibição do float (e.g. \expr{d1.45e-8})
		\end{description}
    \item[\type{Bool}:] \expr{t} para \expr{true}(verdadeiro), \expr{f} para \expr{false}(falso)
    \item[\type{String}:] \expre{y} seguido do comprimento do string codificado para url, seguido de dois pontos e o string codificado em url em si (e.g. \expr{y10:hi\%20there para "hi there".}.
	\item[\type{String}(cacheado):] \expr{R} seguido pelo ID do cache (e.g. \expr{R456}). O cacheamento de string está sempre habilitado.
    \item[Pares nome-valor:]um string serializado representando o nome, seguido pelo valor.
    \item[estrutura:] \expr{o} seguido de uma lista de pares nome-valor, seguidos por \expr{g} (e.g. \expr{oy1:xi2y1:kng} para \expr{\{x:2, k:null\}})
    \item[\type{List}:] \expr{l} seguido da lista dos itens seriados, seguido por \expr{h} (e.g. \expr{lnnh} para uma lista de dois valores nulos (\expr{null}))
    \item[\type{Array}:] \expr{a} seguido por uma lista dos itens seriados, seguido por \expr{h}. Para múltiplos valores \expr{null} consecutivos, \expr{u} segudio pelo número de valores nulos é utilizado (e.g. \expr{ai1i2u4i7ni9h for [1,2,null,null,null,null,7,null,9]})
    \item[Data (tipo\type{Date}):]\expr{v} seguido pela propria data (e.g. \expr{d2010-01-01 12:45:10})
    \item[\type{haxe.ds.StringMap}:] \expr{b} seguido pelos pares nome-valor, seguido por \expr{h}  (e.g. \expr{by1:xi2y1:knh} for \expr{\{"x" => 2, "k" => null\}})
    \item[\type{haxe.ds.IntMap}:] \expr{q} seguido pelos pares chave-valor, seguido por \expr{h}. Cada chave é representada como \expr{:<int>} (e.g. \expr{q:4n:5i45:6i7h} para \expr{\{4 => null, 5 => 45, 6 => 7\}})
    \item[\type{haxe.ds.ObjectMap}:]\expr{M} seguido por pares valores seriados representado a chave e o valor, seguidos por \expr{h}
    \item[\type{haxe.ds.Bytes}:] \expr{s} seguido pela extensão dos bytes codificados em base64, então\expr{:} e a representação de bytes usando os códigos \expr{A-Za-z0-9\%} (e.g. \expr{s3:AAA} para 2 bytes iguais a \expr{0}, \expr{s10:SGVsbG8gIQ} para \expr{haxe.io.Bytes.ofString("Hello !")})
    \item[exception:] \expr{x} seguido do valor da exceção
    \item[{instância de classe:] \expr{c} seguido do nome seriado da classe, seguido do pares nome-valor dos campos, seguido de  \expr{g} (e.g. \expr{cy5:Pointy1:xzy1:yzg} para \expr{new Point(0, 0)} (com dois campos integer \expr{x} and \expr{y})
    \item[instância de enum (por nome):] \expr{w} seguido do nome do enum seriado, seguido pelo nome seriado do constructor, seguido pelo número de argumentos, seguido pelo valor dos argumentos (e.g. \expr{wy3:Fooy1:A0} para \expr{Foo.A} (sem argumentos), \expr{wy3:Fooy1:B2i4n} para \expr{Foo.B(4,null)})
    \item[instância de enum (por índice):] \expr{j} seguido do nome seriado do enum, seguido por \expr{:}, seguido do índice do constructor, seguido pelo número de argumentos, seguido pelo valor dos argumentos (e.g. \expr{wy3:Foo0:0} para \expr{Foo.A} (sem argumentos), \expr{wy3:Foo1:2i4n} para \expr{Foo.B(4,null)})
\item[personalizado(custom):] \expr{C} seguido pelo nome de classe, seguido pelo dado seriado, segudio por \expr{g}
\item[referências ao cache:] \expr{r} seguido pelo índice do cache
\end{description}

\noindent Elementos em cache e constructors de enums são indexados a partir do zero.

\section {Json}
\label{std-Json}

O Haxe oferece suporte interno para a (de)serialização de dados \emph{JSON}\footnote{http://pt.wikipedia/wiki/JSON} via classe \type{haxe.Json}

\subsection{Decomposição sintática de JSON}

Use o método estático \expr{haxe.Json.parse} para decompor sintaticamente dados \emph{JSON} e obter um valor de Haxe deles:

\haxe{assets/JsonParse.hx}

Observe que o tipo de objeto retornado por \expr{haxe.Json.parse} é \expr{Dynamic}, então, se a estrutura de nosso dado é bem conhecida, podemos querer especificar um tipo usando \tref{estruturas anônimas}{types-anonymous-structure}. Dessa maneira, podemos oferecer verificações em tempo de compilação para acessar nossos dados e, provavelmente, melhor geração de código, porque o compilador conhece os tipos em uma estrutura:

\haxe{assets/JsonParseTyped.hx}


\subsection{Codificando JSON}
\label{std-Json-encoding}

Use o método estático \expr{haxe.Json.stringify} para codificar um valor de Haxe em um string \emph{JSON}:

\haxe{assets/JsonStringify.hx}

\subsection{Detalhes de implementação}
label{std-Json-implementation-details}

A API \type{haxe.Json} usa automaticamente a implementação nativa dos targets onde está disponível, i.e., \emph{Javascript}, \emph{Flash} e \emph{PHP} e fornece sua própria implementação para outros targets.

O uso da própria implementação do Haxe pode ser forçado com o argumento de compilação \expr{-D haxeJSON}. Isso também fornece a serialização de \tref{enums}{types-enum-instance} por seus índices, \tref{maps}{std-Map} com chaves string e instâncias de classes.

Browsers mais velhos (Internet Explorer 7, por exemplo) podem não ter implementação nativa de \emph{JSON}. No caso de ser necessário suportá-los, nos podemos incluir uma, dentre as implementações de JSON disponíveis na internet, na página HTML. Alternativamente um argumento de compilação pr{ old_browser} pode ser usado, ele fará \type{haxe.Json} tentar usar o JSON nativo, mas puxa sua própria implementação no caso da nativa ser indisponível.

\section{Xml}
\label{std-Xml}

\section{Input/Output}
\label{std-input-output}

\section{Sys/sys}
\label{std-sys}

\section{Remoting}
\label{std-remoting}

Haxe Remoting é uma forma de fazer comunicação entre duas plataformas diferentes. Com Haxe Remoting, aplicações podem: transmitir dados de maneira transparete, enviar dados e métodos de chamada entre os lados cliente e servidor.

\subsection{Conexão de Remoting}
\label{std-remoting-connection}

Para usar o remoting, deve existir uma conexão estabelecida. Há dois tipos de conexões de Haxe Remoting
\begin{description}
    \item[\expr{haxe.remoting.Connection}] é usada para \emph{conexões síncronizadas}, onde os resultados podem ser diretamente obtidos quando se chama um método.
	\item[\expr{haxe.remoting.AsyncConnection}] é usado para \emph{conesões assíncronas}, onde os resultados são eventos que acontecerão posteriormente no processo de execução.
\end{description}

\paragraph{Inicie uma conexão}
Existem alguns constructors específicos de targets com propósitos diferentes, que podem ser usados para estabelecer uma conexão:

\begin{description}
	\item[Todos os targets:]
		\begin{description}
			\item[\expr{HttpAsyncConnection.urlConnect(url:String)}]  
				Retorna para a URL dada uma conexão assíncrona que deve ligar a uma aplicação servidora do Haxe 
		\end{description}
		
	\item[Flash:]
		\begin{description}
			\item[\expr{ExternalConnection.jsConnect(name:String, ctx:Context)}]  
			    Permite uma conexão ao código de Haxe Javascript local. O código Haxe JS deve ser compilado com a classe ExternalConnection incluída. Isso só funciona com Flash Player 8 e acima
			\item[\expr{AMFConnection.urlConnect(url:String)} e \expr{AMFConnection.connect( cnx : NetConnection )}]  
                Permite uma conexão a um \href{http://en.wikipedia.org/wiki/Action_Message_Format}{AMF Remoting server} tal como um  \href{http://www.adobe.com/products/adobe-media-server-family.html}{Flash Media Server} ou \href{http://www.silexlabs.org/amfphp/}{AMFPHP}.
			\item[\expr{SocketConnection.create(sock:flash.XMLSocket)}]  
                Permite comuincações de remoting sobre um \type{XMLSocket}
			\item[\expr{LocalConnection.connect(name:String)}]  
				Permite comunicações remoting sobre uma \href{http://api.haxe.org/haxe/remoting/LocalConnection.html}{Flash LocalConnection}
		\end{description}
	\item[Javascript:]
		\begin{description}
			\item[\expr{ExternalConnection.flashConnect(name:String, obj:String, ctx:Context)}]  
				Permite a conexão entre um dado Objeto Flash. O conteúdo Haxe Flash deve ser carregado e tem que incluir a classe \expr{haxe.remoting.Connection} class. Isso só funciona com Flash 8 ou maior 
		\end{description}
		
	\item[Neko:]
		\begin{description}
			\item[\expr{HttpConnection.urlConnect(url:String)}]  
                Funcionará como a versão assíncrona mas em modo sincronizado.
			\item[\expr{SocketConnection.create(...)}]  
                Permite comunicações em tempo real com um cliente Flash que esteja usando um \type{XMLSocket} para conectar ao servidor
		\end{description}
\end{description}

\paragraph{Remoting context}

Antes de se fazer a comunicação entre plataformas, um contexto de remoting (remoting context) tem que ser definido. Isso é uma API compartilhada que pode ser chamada pelo código do cliente na conexão.

Esse exemplo de código servidor cria e compartilha uma API

\begin{lstlisting}
class Server {
	function new() { }
	function foo(x, y) { return x + y; }

	static function main() {
		var ctx = new haxe.remoting.Context();
		ctx.addObject("Server", new Server());
		
		if(haxe.remoting.HttpConnection.handleRequest(ctx))
		{
			return;
		}
		
		// handle normal request
		trace("This is a remoting server !");
	} 
}
\end{lstlisting}

\paragraph{Usando a conexão}

Usar uma conexão é bem conveniente. Uma vez que a conexão pe obtidad, use o clássico acesso-por-ponto para atribuir valor a um caminho e então use a \expr{call()} para chamar o método no contexto remoto e pegar o resultado.
A conexão assíncrona toma uma parâmetro de função adicional que será chamado quando o resultado estiver disponível.

Esse exemplo de código cliente se conecta ao contexto de remoting servidor e chama uma função \expr{foo()} de sua API.
\begin{lstlisting}
class Client {
  static function main() {
    var cnx = haxe.remoting.HttpAsyncConnection.urlConnect("http://localhost/");
    cnx.setErrorHandler( function(err) trace('Error: \$err'); } );
    cnx.Server.foo.call([1,2], function(data) trace('Result: \$data'););
  }
}
\end{lstlisting}

Para fazer isso funcionar para o target Neko, prepare um Neko Web Server, aponte a url no cliente para \ic{"http://localhost2000/remoting.n"} e compile o servidor usando \ic{-main Server -neko remoting.n}.

\paragraph{Manuseio de erros}

\begin{itemize}
    \item Quando um erro ocorre em uma chamada assíncrona, o manuseador de erro (error handler) é chamado como visto no exemplo acima.
	\item Quando um erro ocorre em uma chamada sincronizada, uma excessão é levantada do lado onde a chamada foi feita como se estivéssemos chamando um método local. 
\end{itemize}

\paragraph{Data serialization}

O Haxe Remoting pode mandar muitos tipos diferentes de dados. Veja \tref{Serialização}{std-serialization}.

\subsection{Detalhes de Implementação}
\label{std-remoting-implementation-details}

\paragraph{Especificidades de segurança de Javascript}

A página-html envelopando o cliente js deve ser servido do mesmo domínio de onde o servidor está executando. A política de memsa origem restringe como um documento ou script carregados da mesma oirgem podem interagrir com um recurso de outra origem. A política de mesma origem é usada como uma maneira de prevenir alguns dos ataques de falsificação de requisição cruzada de sites.

Para usar o remoting entre fronteiras de domínios, CORS (cross-origin resource sharing) precisa ser habilitado com a definição do cabeçalho \ic{X-Haxe-Remoting} no \ic{.htaccess}:

\begin{lstlisting} 
# Enable CORS
Header set Access-Control-Allow-Origin "*"
Header set Access-Control-Allow-Methods: "GET,POST,OPTIONS,DELETE,PUT"
Header set Access-Control-Allow-Headers: X-Haxe-Remoting
\end{lstlisting} 

Veja  \href{http://pt.wikipedia.org/wiki/Pol\%C3\%ADtica_de_mesma_origem}{política de mesma origem} para mais informação sobre esse tópico.

Também observe que isso significa que a página não pode ser servida diretamente do sistema de arquivos \ic{"file:///C:/example/path/index.html"}.

\paragraph{Especificidades de segurança de Flash}

Quando o Flash acessa um servidor de um domínio diferente, prepare um arquivo \ic{crossdomain.xml} no servidor, habilitando os cabeçalhos \ic{X-Haxe}.

\begin{lstlisting} 
<cross-domain-policy>
	<allow-access-from domain="*"/> <!-- or the appropriate domains -->
	<allow-http-request-headers-from domain="*" headers="X-Haxe*"/>
</cross-domain-policy>
\end{lstlisting} 

\paragraph{Tipos de argumentos não são assegurados}

Não há garantia de nenhuma espécie que os tipos dos argumentos serão respeitados quando um método é chamado usando remoting.
Isso quer dizer que mesmo se os argumentos da função \expr{foo} são tipados para \type{Int}, o cliente ainda será capaz de usar strings quando chamar o método.
Isso pode levar a questões de segurança em alguns casos. Quando em dúvida, verifique o tipo do argumento ao chamar a função, usando o método \expr{Std.is}.


\part{Miscellaneous}
% \chapter{Haxelib}
\label{haxelib}

Haxelib é o gerenciado de bibliotecas que vem com qualquer distribuição do Haxe. Conectado a um repositório central, ele permite submeter e retirar bibliotecas e tem múltiplas funcionalidades além dessa. Bibliotecas disponíveis podem ser encontradas em \url{http://lib.haxe.org}.

Uma biblioteca de Haxe é uma coleção de arquivos \ic{.hx}. Isso é, bibliotecas são distribuídas pelo código fonte por padrão, tornando fácil inspecionar e modificar seu comportamento. Cada biblioteca é identificada por um nome único, que é utilizado quando se informa o compilador do Haxe quais bibliotecas usar em uma dada compilação.

\section{Usando a bilblioteca do Haxe com o compilador do Haxe}
\label{haxelib-using-haxe}

Qualquer biblioteca instalada do Haxe pode ser disponibilizada para o compilador através do argumento -lib<nome-da-biblioteca>. Isso é muito similar ao argumento \ic{-cp<path>}, mas espera um nome de biblioteca ao invés de um caminho de diretório. Esses comandos são explicados detalhadamente em \Fullref {compiler-usage}.

Para nosso uso exemplificativo, escolhemos uma biblioteca muito simples de Haxe, chamada ``random''. Ela oferece um conjunto de métodos estáticos convenientes para conseguir diversos efeitos aleatórios (randômicos), como escolher um elemento de um array.

\haxe{assets/HaxelibRandom.hx}

A compilação disso sem qualquer argumento \ic{-lib} gera uma mensagem de erro \ic{Unknow identifier : Random} ao longo das linhas. Isso mostra que as bibliotecas instaladas não estão disponíveis para o compilador por padrão a não ser que sejam explicitamente adicionadas. Uma linha de comando que funcione para o programa acima é \ic{haxe -lib random -main Main --interp .

    Se o compilador enviar um erro \ic{Error: Library random is not installed: run 'haxelib install random'} a biblioteca tem que ser instalada via comando \ic{haxelib} primeiro. Como a mensagem de erro sugere, isso é conseguido através do comando \ic{haxelib install random}. Aprenderemos mais sobre o comando haxelib em \Fullref{haxe-lib-using}.

\subsection{haxelib.json}
\label{haxelib-json}

Cada biblioteca do Haxe exige um arquivo \ic{haxelib.json}, onde os seguintes atributos são definidos:

\begin{description}
    \item[name:] O nome da biblioteca. Deve conter pelo menos três caracteres entre os seguintes: \ic{\[A-Za-z0-9_-.\]}. Em particukar, espaços não são permitidos.
    \item[url:] A url da biblioteca, i.e., onde mais informação pode ser localizada;
    \item[license:] A licença sob a qual a biblioteca é publicada. Pode ser \ic{GPL}, \ic{LGPL}, \ic{BSD}, \ic{Public} (para domínio público) ou \ic{MIT}.
    \item[tags:] um array de strings com rótulos que são usados para o website do repositório ordenar as bibliotecas.
    \item[descrição:] A descrição do que á biblioteca está fazendo.
    \item[versão:] Um string de versão da biblioteca. Isso é detalhado em \Fullref{haxelib-json-versioning}.
    \item[releasenote:] Observações da publicação da versão em questão.
    \item[contributors:] Um array de nomes de usuários que identifique os contribuídores para a biblioteca.
    \item[dependencies:] Um objeto descrevendo as dependências da biblioteca. Isso é detalhado na seção \Fullref{haxelib-json-dependencies}.
\end{description}

A JSON seguinte é um simples exemplo de um haxelib.json:

\begin{lstlisting}
{
    "name": "useless_lib",
    "url" :
        "https://github.com/jasononeil/useless/",
    "license": "MIT",
    "tags": ["cross", "useless"],
    "description":
        "This library is useless in the same way on
         every platform",
    "version": "1.0.0",
    "releasenote":
         "Initial release, everything is working
          correctly",
    "contributors": ["Juraj","Jason","Nicolas"],
    "dependencies": {
        "tink_macros": "",
         "nme": "3.5.5"
     }
}
\end{lstlisting}

\subsection{Versionamento}
\label{haxelib-json-versioning}

A Haxelib usa uma versão simplificada do \href{(http://semver.org)}{SemVer}. O formato básico é este:

\begin{lstlisting}
major.minor.patch
\end{lstlisting}

As regras básicas são:

\begin{itemize}
    \item Versões major são incrementadas quando se quebra compatibilidade retroativa - de forma que códigos antigos não funcionarão com a nova versão da biblioteca.
    \item Versões minor são incrementadas quando novas funcionalidades são adicionadas
    \item Versões patch são para pequenas correções que não alteram a API pública. 
    \item Quando uma versão menor é incrementada, o número patch é retornado para 0. Quando uma versão major é incrementada, tanto o número minor quanto o número patch são retornados para zero.
\end{itemize}

Exemplos:

\begin{description}
    \item [0.0.1:] Uma primeira divulgação. Qualquer coisa com um 0 para versão major pe sujeito a mudanças na próxima divulgação - nenhuma promessa de estabilidade de API!
    \item [0.1.0:] Inclui uma nova funcionalidade! Incrementa a versão minor, zera a versão patch
    \item [0.1.1:] Percebeu-se que a nova funcionalidade estava quebrada. Consertada agora, se incrementa o número patch
    \item [1.0.0:] Nova versão major, assim se incrementa o número major e se zera os números minor e patch. Você promete aos seus usuários não quebrar essa API até que você pule para 2.0.0
    \item [1.0.1:] Um conserto menor
    \item [1.1.0:] Uma nova funcionalidade
    \item [1.2.0:] Outra nova funcionalidade
    \item [2.0.0:] Uma nova versão, que pode quebrar a compatibilidade com 1.0. Usuários devem ser cuidadosos ao se atualizarem.
\end{description}


Se essa divulgação é um pré-visão (Alfa, Beta ou Candidato a lançamento), você pode incluir iss, com um número opcional de divulgação:

\begin{lstlisting}
major.minor.patch-(alpha/beta/rc).release
\end{lstlisting}

Exemplos:

\begin{description}
    \item[1.0.0-alpha:] O alfa de 1.0.0 - use com cuidado, as coisas estão mudando
    \item[1.0.0-alpha.2:] O segundo alfa
    \item[1.0.0-beta:] Beta - as coisas estão se firmando, mas ainda sujeitas a alterações.
    \item[1.0.0-rc.1:] O primeiro candidato a lançamento (rc=release candidate) para 1.0.0 - não se colocarão novas funcionalidades agora
    \item[1.0.0-rc.2:] O segundo candidato a lançamento para 1.0.0
    \item[1.0.0:] o lançamento final!
\end{description}

\subsection{Dependências}
\label{haxelib-json-dependencies}

A partir do Haxe 3.1.0, a haxelib suporta unicamente o casamento de versões exatas como dependências. Dependências são definidas como parte do \tref{haxelib.json}{haxelib-json} com o nome da biblioteca servindo como chave e a versão esperado (se requerido) como um valor no formato descrito em \Fullref{haxelib-json-versioning}.

Já vimos um exemplo disso quando apresentamos o haxelib.json:

\begin{lstlisting}
"dependencies": {
    "tink_macros": "",
    "nme": "3.5.5"
}
\end{lstlisting}

Isso adiciona duas dependências para a dada biblioteca Haxe:

\begin{enumerate}
    \item A biblioteca ``tink.macros'' pode ser usada em qualquer versão. Haxelib tentará, entãom sempre usar a última versão.
    \item A bibiblioteca ``nme'' é necessária na versão ``3.5.5''. A Haxelib fará certo que essa exata versão seja usada, evitando potenciais mudanças drásticas com versões futuras.
\end{enumerate}

\section{extraParams.hxml}
\label{haxelib-extraParams}

Se você somar um arquivo chamado \ic{extraParams.hxml} a raiz da sua biblioteca (no mesmo nível que \ic{haxelib.json}) esses parâmetros serão automaticamente somados aos parâmetros de compilação quando alguém usar sua biblioteca com \ic{-lib}.

\section{Usando a Haxelib}
\label{haxelib-using}

Se o comando \ic{haxelib} é executado sem quaisquer argumentos, ele imprime uma lista exaustiva dos argumentos disponíveis. Acesse o website \url{http://lib.haxe.org} para ver todas as bibliotecas disponíveis.

Os seguintes comandos estão disponíveis

\begin{description}
	\item[Básicos]
		\begin{description}
			\item[\ic{haxelib install [nome-do-projeto] [versão]}] instala o projeto mencionado. Você pode opcionalmente especificar uma versão para ser instalada.
			\item[\ic{haxelib update [nome-do-projeto]}] atualiza uma biblioteca singular para a sua última versão. 
			\item[\ic{haxelib upgrade}] eleva todos os projetos instalados para suas últimas versões. Esse comando demanda uma confirmação para cada projeto que pode ser atualizado.
			\item[\ic{haxelib remove nome-do-projeto [versão]}] removerá um projeto completo ou unicamente uma versão, se especificada.
			\item[\ic{haxelib list}] lista todos os projetos instalados e suas versões. Para cada projeto a versão entre colchetes é a atual.
			\item[\ic{haxelib set [nome-do-projeto] [versão]}] muda a versão atual de um dado projeto. A versão já deve estar instalada.
		\end{description}
		
	\item[Informativos]
		\begin{description}
			\item[\ic{haxelib search [palavra]}] lista os projetos que tem ou um nome ou uma descrição que coincida com a palavra especificada.
			\item[\ic{haxelib info [project-name]}] will give you information about a given project.
			\item[\ic{haxelib user [nome-de-usuário]}] lista informação sobre um dado usuário da  Haxelib.
			\item[\ic{haxelib config}] retorna o caminho do repositório da Haxelib. Isto é onde a Haxelib fica instalada por padrão.
			\item[\ic{haxelib path [nome-do-projeto]}] retorna o caminho para as bibliotecas e suas dependecias (definidas em \ic{haxelib.xml}).
		\end{description}
		
	\item[Para desenvolvimento]
		\begin{description}
			\item[\ic{haxelib submit [projeto.zip]}] submete um pacote à HaxeLib. Se o nome do usuário é desconhecido, você será primeiro solicitado a registrar uma conta. Se o usuário já existir, você será questionado por sua senha. Se o projeto não existe ainda, ele será cirado, mas nenhuma versão será adicionada. Você terá que submetê-la uma segunda vez para adicionar a primeira versão divulgada. Se você quiser mudar o url do projeto ou sua descrição, simplesmente modifique seu \ic{haxelib.xml} (mantendo a informação de versão inalterada) e submita novamente.
			\item[\ic{haxelib register [nome-do-projeto]}] submete ou atualiza um pacote de biblioteca.
            \item[\ic{haxelib local [nome-do-projeto]}] testa o pacote da biblioteca. Garanta que tudo (tanto a instalação quanto o uso) estão funcionando corretamente antes da submissão, visto que uma vez submetida, uma dada versão não pode ser atualizada.
			\item[\ic{haxelib dev [nome-do-projeto] [diretório]}] definirá um diretório de desenvolvimento para o dado projeto. Para definir o diretório do projeto devolta ao loal globra, rode o comando e omita o diretório.
            \item[\ic{haxelib git [nome-do-projeto] [caminho-para-clonar-no-git] [branch-do-git] [subdiretório]}] usa um repositório do git como biblioteca. Isso é útil para usar uma versão de desenvolvimento mais atual, uma variante(fork) do projeto original, ou para ter uma biblioteca particular que você não deseja publicar na Haxelib. Quando você usa \ic{haxelib upgrade}, quaisquer bibliotecas que são instaladas usando GIT puxarão automaticamente a última versão.
        \end{description}
		
	\item[Outros]
		\begin{description}
			\item[\ic{haxelib setup}] define o caminho para o repositório. Para ver o atual, use \ic{haxelib config}.
			\item[\ic{haxelib selfupdate}] atualiza a própria Haxelib. Solicitará a execução de \ic{haxe update.hxml} depois dessa atualização.
			\item[\ic{haxelib convertxml}] converte o arquivo \ic{haxelib.xml} para \ic{haxelib.json}.
			\item[\ic{haxelib run [nome-do-projeto] [parametros]}] executa a biblioteca específicada com parâmetros. Exige um arquivo Haxe/Neko \ic{run.n} précompilado no pacote da biblioteca.Isso é útil se você quer que usuários sejam capazes de fazer algum script para rodar após a instalação que configurará coisas adicionais no sistema. Seja cuidadoso em relação a confiabilidade do projeto que você está rodando, uma vez que o script pode danificar seu sistema.
			\item[\ic{haxelib proxy}] define o proxy de Http.
		\end{description}
\end{description}

% \input{12-target-details.tex}

\end{document}
